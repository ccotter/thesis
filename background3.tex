
\section{Background}

With an understanding of the goal of this thesis, we will now discuss the
benefits of deterministic execution. Then, we will present the Determinator
kernel and user library design.

\subsection{Benefits of determinism}
\label{sec:det-motiv}

Determinism provides many benefits to application
developers~\cite{Bergan11,olszewski2009kendo,bocchino2009parallel}. Bergan et
al. suggest there are four main benefits in
the following areas: debugging, fault-tolerance, testing, and security.

\paragraph{Debugging} Debugging multithreaded programs can be difficult, because
often bugs are not easily reproducible. Tools such as {\tt gdb} are not
always useful for tracking down heisenbugs~\cite{Musuvathi08}. Finding a bug's
root cause becomes easier when a program's execution can be replayed.

\paragraph{Fault-tolerance} Fault tolerance through replication relies
on the assumption that running a program multiple times will always return the
same output. Repeatability is a natural benefit of determinism.

\paragraph{Testing} The difficulties in testing multithreaded applications are
compounded by racy nondeterministic scheduling. Developers and automated test
systems must consider the exponential blow up of possible scheduling sequences.
Determinism helps alleviate this problem, since for each input, there is exactly
one possible logical scheduling sequence of threads. This observation eliminates
the need to consider what scheduling interactions can occur and ultimately
helps developers design test strategies~\cite{Bergan11}.

Since schedule sequences may still be arbitrarily deterministic, developers may
still have a hard time designing test suites. Predictable programming models
like Determinator allow developers to reason about code beforehand
to design a more sophisticated testing strategy.

\paragraph{Security} Processes sharing a CPU or other hardware should be
conscious about leaking sensitive data. A malicious thread can exploit covert
timing channels to extract sensitive data from other, perhaps privileged,
threads~\cite{Aviram10cloud}. Determinism eliminates covert timing channels,
since a program is purely a function of explicit inputs and cannot possibly rely
subtly on the timings of hardware operations.
\\

\iffalse
We also note the importance of repeatability in simulators.
Physicists rely on repeatability of simulated experiments to verify results,
and video game developers often benefit from repeatable physics engines.
\fi

Whereas individual tools like record and replay debuggers aid programmers in
single areas, these so called ``point solutions\ldots do not compose well with
one another''~\cite{Bergan11}. On the other hand, determinism provides benefits
in all four areas with a single mechanism without any overhead besides that
inherent in the deterministic environment itself.

\iffalse
To further motivate determinism, we consider ``point solutions'' that
solve problems in single areas at
once. Record and replay debuggers, like Leblanc et al.'s Instant Replay
system, aid in debugging parallel programs by logging scheduling sequences and
other relevant interactions in order to replay an execution sequence exactly.
However, these debuggers are costly in terms of storage and performance.
\iffalse Even with replay ability, the execution sequence is still arbitrary and
gives the programmer no intuition about the program's behavior. \fi
In general, these ``point solutions\ldots do not compose well with one
another''~\cite{Bergan11}. On the other hand, determinism provides benefits in
all four areas at once with a single mechanism without any additional overhead
besides that inherent in the deterministic environment.
\fi

\subsection{Determinator}

Aviram et al. set out to provide
\begin{quote}
a parallel environment that:
(a) is ``deterministic by default,'' except when
we inject nondeterminism explicitly via external inputs;
(b) introduces no data races, either at the memory access level
or at higher semantic levels; (c)
can enforce determinism on arbitrary, compromised or
malicious code for security reasons; and (d) is efficient
enough to use for ``normal-case'' execution of deployed
code, not just for instrumentation during development. \cite{Aviram10}
\end{quote}

To this end, they presented Determinator, a novel OS written from the ground up.
For most of the remainder of this section, we will recapitulate Aviram et al.'s
work and contributions. First we will discuss aspects that influenced
Determinator's design. Then, we will look at the actual kernel design itself
and the accompanying user library.

The primary cause of nondeterminism is data races introduced by timing
dependencies. Each source of implicit nondeterminism must be accounted for in
designing a deterministic programming model. We discuss them here, and describe
how Determinator handles them.

\paragraph{Explicit nondeterminism}
Often, programs rely on semantically relevant nondeterministic inputs such as
network packets, user input, or clock time. A deterministic programming model
must incorporate these inputs while still enforcing determinism. Determinator
addresses these inputs by turning them into explicit I/O~\cite{Aviram10}.
Applications have complete control over these input sources and can log the
inputs for reply debugging.

\paragraph{Shared program state}
Traditional multithread programming models provide shared state: processes share
the system's file system and threads using the pthreads API share the entire
memory state. Data races and buggy application of synchronization primitives
lead to nondeterministic execution traces and introduce unpredictable bugs.

Determinator eliminates data races caused by shared program state by eliminating
shared state altogether. Threads operate using a \emph{private workspace model}
and synchronize program state at explicitly defined program
points~\cite{Aviram10}. When two or more threads begin executing, each has
identical private virtual memory images. Writes to memory are not visible to
other threads until the threads explicitly synchronize.

\paragraph{Nondeterministic scheduling abstractions}
Traditional multithreaded synchronization abstractions are often neither
deterministic nor predictable. Random hardware races determine the next thread
to acquire a mutex lock, and as mentioned in section \ref{sec:det-motiv}, this
has debugging and testing implications. Even though we can record lock
acquisition sequences to replay program execution or use some arbitrary device
to choose a deterministic sequence, the order of acquisition is not predictable.

Determinator restricts itself to only allow naturally deterministic and
predictable synchronization abstractions such as
fork-join~\cite{nelson1988approximate}. In the fork-join paradigm, threads
\emph{fork} children threads to perform some computation. The original thread
gathers the results by doing a \emph{join} by blocking until the child thread
completes.

\paragraph{Globally shared namespaces}
Operating systems introduce nondeterminism by using namespaces that are shared
by the entire system. Process IDs returned by {\tt fork()} and files created
by {\tt mktemp()} are examples. Since these identifiers are nondeterministic,
and only the resource itself, not the identifier, is semantically relevant to
the application, Determinator disallows the system from choosing resource
identifiers from globally shared namespaces. Instead, applications themselves
choose identifiers deterministically. For example, when performing a
{\tt fork()}, the user program must specify the child ID as an argument instead
of the operating system returning the process ID of the newly created child.

\subsection{The Determinator kernel}

Determinator organizes processes in a nested process
model~\cite{Ford96,Aviram10}.
Processes cannot outlive their parents and can communicate only with their
parent and children. The kernel ``provides no file systems, writable shared
memory, or other abstractions that imply globally shared state''. Only ``the
distinguished root [process] has direct access to nondeterministic
inputs''~\cite{Aviram10}. It is this root process that can control explicitly
nondeterministic inputs like network packets. All other processes must
communicate directly or indirectly with the root process to access I/O devices.

Processes execute in a \emph{private workspace} model. Each process (or thread)
keeps a private copy of the application's virtual memory. Processes do not
see another process's writes until reconciliation at explicit synchronization
points. This model eliminates read/write races, and write/write races are still
deterministic since the program defines reconciliation order.

\input{tbl1}
\input{tbl2}

\paragraph{Kernel interface} Processes communicate with the kernel via three
syscalls, {\tt Put}, {\tt Get}, {\tt Ret}. Table \ref{tab:syscalls} and Table
\ref{tab:options}, reproduced from~\cite{Aviram10}, summarize how the syscalls
work and the options available to {\tt Put} and {\tt Get}.

Determinator enforces a deterministic schedule by requiring programs to
explicitly define synchronization points. Since the kernel does not manage any
global namespaces, user programs specify
a child process ID parameter to {\tt Put} and {\tt Get}. The first {\tt Put}
syscall with a previously unused child ID creates a new child process.
{\tt Put} calls can start a child's execution (via the {\tt Start} flag), and
the child will continue to execute until it issues a {\tt Ret}. Processor
exceptions (e.g. divide by zero) generate an implicit {\tt Ret} that must be
acknowledge by the parent. Both {\tt Put} and {\tt Get} calls block until the
child process stops.

Process state is composed of its register state and virtual memory image.
The {\tt Regs} option copies register state from a parent to child or vice
versa. The {\tt Zero} option zeros a virtual memory region in a child. The
{\tt Copy} option copies virtual memory into or out of a child.

Determinator provides a more sophisticated memory utility: snapshot-merge.
{\tt Snap} copies the calling process's entire virtual memory state into the
specified child. Invoking {\tt Get} with the {\tt Merge} parameter performs a
three-way diff and merge. The kernel compares bytes
in the child that have changed since the previous {\tt Snap} invocation. Bytes
that have changed in the child only are copied into the parent. Bytes that
changed in the parent but not the child are not copied.
Bytes that changed in both the parent and child generate a
\emph{merge conflict} exception. The kernel implements {\tt Merge} efficiently
by examining page table entires. This snapshot-merge mechanism is useful for
providing applications with an easy way to implement fork-join. A process forks
many children to perform some computation and simply {\tt Merge} the results
back into itself.

Aviram et al. conclude their discussion of Determinator's kernel by mentioning
that the three syscall primitives ``reduce to blocking, one-to-one message
channels, making the space hierarchy a deterministic Kahn
network''~\cite{Aviram10,kahn1974semantics}.

\subsection{Determinator's user library}

The Determinator kernel alone is enough to enforce deterministic program
execution; to make writing deterministic programs more natural, however, Aviram
et al. provide a high-level user library that wraps around the three syscall
interface. In this section, we will go over
the five main areas discussed by Aviram et al. in their ``Emulating High-Level
Abstractions'' section: process API, file system, I/O, shared memory
multithreading, and legacy thread APIs~\cite{Aviram10}.

\paragraph{Process API}
Determinator provides an interface similar to that of {\tt fork/exec/wait}.
All of these functions are implemented in user space instead of kernel space.
To {\tt fork()} a child process, the parent invokes {\tt dput()} to copy its
register and memory state into a new child. The user library must manage a list
of ``free'' process ID numbers, because the system itself does not manage
process IDs. {\tt waitpid()}
works by entering a loop querying the status of the child; if the child needs
more input to continue running (through a mechanism described below), it
sets its status appropriately and issues a {\tt dret()}. The parent gives the
child more input and sets it in motion again. Once the child finishes executing,
it marks itself done and issues a {\tt dret()}. The parent collects the child's
status and kills the process.

{\tt exec()} works by forking a child process and loading the new program
the new program's memory image into the child. This child is never actually run.
Instead, {\tt exec()} enters a trampoline code segment that does a {\tt Get} to
copy the new program into the existing process. The trampoline code is mapped at
identical locations in both processes so that after executing the {\tt Get}, the
process begins executing valid code.

\paragraph{File system}
Since processes can only access their register set and memory, Determinator
provides an in-memory file system. Each process has a private copy of the
system's file system. A {\tt fork()} copies the parent's file system state into
the child. The parent and child work on private copies of the file system and
merge their changes at synchronization points using file versioning
techniques~\cite{parker1983detection}. Two files may not be
concurrently modified; such cases lead to a reconciliation conflict. The parent
and child may, however, perform \emph{append-only} changes to the same file.
The file is reconciled by appending the child's additions to the end of the
parent's file, and vice-versa.

The file system has limitations compared to traditional file systems.
The total file system size is limited
by the process's address space; on 32-bit systems, this is a serious limitation.
Since the file system resides in virtual memory, buggy programs can write to
the memory where the file system resides, corrupting the file system. Lastly,
Determinator's implementation of the file system only supports up to 256 files,
each with a max of 4MB in size~\cite{Aviram10cloud}.

\paragraph{I/O}
Since Deterministic processes have no access to external I/O, Determinator
emulates I/O as a special case of the file system. Library functions like
{\tt getchar()} and {\tt printf()} read and write from special files
{\tt stdin} and {\tt stdout}, respectively.

A {\tt printf()} appends output to {\tt stdout}. When a parent merges its
file system with a child, {\tt stdout} output is forwarded
to the parent and eventually reaches the root process where the root can
actually write the output to the system's I/O device.
A program does a {\tt read()} to obtain the next unread character(s) in
{\tt stdin}. If the file is out of unread characters, {\tt read()} issues a
{\tt dret()} to ask for more input from the parent.

Since the file system supports ``append-only'' conflicts, the above strategy
works well for handling I/O. As all processes reconcile their file
systems, each process will see all other process's {\tt printf()}s.

\paragraph{Legacy multithreading APIs}
Determinator can emulate shared memory multithreading and other legacy
multithreaded APIs like pthreads. However, we will not discuss either here,
since we do not use these techniques in deterministic Linux. However, we note
that since Determinator emulates these legacy thread APIs using its three
syscall interface, deterministic Linux could very well be extended to support
these APIs. The reader is referred to Aviram et al.'s sections 4.4 and
4.5~\cite{Aviram10}.

\endinput

