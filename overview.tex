\section{Overview}

We begin the discussion of adding determinism to Linux by discussing overall
design goals of the project. Next we look at the challenges of making
nondeterministic Linux deterministic. For the rest of this thesis, we
distinguish between \emph{legacy} Linux (the unmodified Linux kernel) and
\emph{deterministic} Linux.

\subsection{Design Goals}

We wish to make 64-bit Linux deterministic, and in doing we will present an
interface similar to that of Determinator. We also would like to run legacy
Linux applications alongside deterministic applications, for this is one of the
motivating factors applying Determinator's design to Linux. We want to run
legacy applications without modification, but we won't make any attempt to force
legacy applications to run deterministically. Legacy applications will run in a
legacy nondeterministic mode. In order to take advantage of determinism in
Linux, legacy programs must be rewritten using the new operating system
abstractions.

We would also like to write a user level C library with familiar abstractions
such as fork-join and an in memory file system based on those of Determinator.
In some cases, we improve upon Determinator's user library, especially the
limitations of the in memory file system~\cite{Aviram10,Aviram10cloud}.

We do not wish to apply all of Determinator's features to a deterministic Linux.
Determinator supports deterministic distributed cluster computing by extending
its nested process model to a cluster of nodes. Determinator also supports a
``tree'' copy operation for {\tt Put} and {\tt Get}. Lastly, Determinator allows
threads to place an instruction limit on children threads. We have no intention
of supporting these features, but we note this limitation does not detract from
making Linux deterministic.

Since the primary goal is to make Linux deterministic, we may decide
to limit or ignore features of the Linux kernel internals. For example, Linux
supports huge pages of memory alongside ``normal'' 4-KB pages. Since this is an
internal optimization that is hidden from user applications and for reasons of
implementation complexity, we may not allow deterministic applications to take
advantage of certain kernel features. We would like to keep all existing
functionality available to applications running in legacy mode, however.

Moreover, some useful features may be unavailable to deterministic processes.
For example, shared dynamic libraries require nontrivial support from the Linux
kernel and standard C library. There is nothing limiting us from devising ways
to support features like this, but we feel it is out of scope of our primary
goal.

\iffalse
\begin{itemize}
	\item Design Goals/Non-goals
	\begin{itemize}
		\item Make Linux x86\_64 deterministic by adopting Determinator's OS design.
		\item Legacy Linux programs will behave as usual, but deterministic Linux
			programs might only support a subset of Linux's features (e.g. no huge page
			tables).
		\item Make no attempt to replicate Determinator's distributed computing
			aspects.
		\item Write C user library with familiar abstractions (e.g. fork/join) and
			in memory file system, again based on that of Determinator. This user library
			is a layer on top of Determinator's three syscall interface.
		\item Runtime environment features might be reduced for deterministic programs
			(e.g. no shared library support, won't support standard C library
			functions).
	\end{itemize}
\fi

\subsection{Challenges}

We have already discussed the four categories of nondeterminism identified by
Aviram et al; these obvservations are general enough that they also apply in
making Linux deterministic. In adapting Determinator's design to Linux, however,
we must address the following additional issues.

\iffalse
capitalization in paragraph titles?
\fi
\paragraph{Inherent nondeterminism in Linux}
In order to run legacy Linux applications, we cannot enforce that all but a
single root process operate in deterministic mode; this design aspect must be
reexamined to allow legacy and deterministic applications to run side-by-side.
Furthermore, Linux's process model allows reparenting and for children to
outlive parents, directly opposed to Determinator's nested process model.

Linux's threading model is inherently nondeterministic and provides many
additional sources of nondeterminism than those already addressed by the above
discusion: Linux supports signals and System V IPC. To address most sources of
nondeterminism (implicit and explicit), Determinator's designers simply did not
add support for these features, since Determinator was written from scratch. On
the other hand, Linux already provides extensive support for nondeterministic
features (e.g. the {\tt gettimeofday()} syscall).

\paragraph{Memory subsystem} Linux supports a wide range of virtual memory
features including memory mapped files, huge pages, and swapping to disk; all
of these features are layered on top of an abstraction for supporting memory
management units for a wide range of processor types. Compared to Determinator,
Linux's memory subsystem uses much more complicated abstractions to support
these features. Memory operations (Zero, Copy, and Snap/Merge) are central to
Determinator's design, so understanding and overcoming this complicated system
is essential to implementing determinism in Linux.

\iffalse
	\item Challenges
	\begin{itemize}
		\item Address four sources of non-determinism identified by Aviram.
			Determinator addressed these by not adding these features in its OS
			design. Linux already supports these features, so they must be removed
			or restricted.
		\begin{itemize}
			\item ``Semantically-relevant non-deterministic'' inputs should be made
			into controllable explicit I/O.
			\item Programs cannot share state (memory).
			\item Non-deterministic scheduling abstractions: data racy locks must
				be disallowed.
			\item Globally shared namespaces (process IDs) must not be used.
		\end{itemize}
		\item By design, Determinator enforces all but the root process to be
			deterministic, so we must rethink this design aspect.
		\item We would like to support legacy Linux programs (we don't want to
			have to rewrite init and every other popular Linux program).
		\item Linux process model allows reparenting and children to outlive
			parents, must be disallowed. Processes have more complicated relationships:
			thread ID, process ID, group ID, session ID.
		\item Linux supports signals, System V IPC, disk-backed file system which must
			be disabled for deterministic processes.
		\item Memory subsystem uses more complicated abstractions for managing a
			process's memory: list of contiguous memory regions with complicated
			rules, generic four level page table, non-trivial locking rules, and
			disk-swappable private pages. Also supports huge pages (2MB) and page
			deduping ``kernel samepage merging'' - these all complicate memory
			management considerations.
		\item Specific example: copy-on-write can only be employed when two mm\_structs
			have vm\_area\_regions with matching start and end addresses. Reason: page
			frame reclaiming assumes a page has the same offset from the vm\_area\_region
			start address in every page table entry.
\fi

\paragraph{Standard C Library} Many Linux applications written in C use
the standard C library. This library in large part functions as a wrapper around
legacy Linux syscalls, and thus is highly nondeterministic. Whereas some
functions, such as {\tt strlen()} might not use nondeterministic syscalls, many
other functions do use nondeterministic syscalls (e.g. {\tt printf()}). Thus,
deterministic programs may be forced to use a completely different library.
Moreover, the libraries in Linux are often linked dynamically with shared
libraries, but Determinator does not provide any native kernel support for
dynamic linking. We may lose the ability to dynamically link shared libraries.

\iffalse
		\item Standard C library implementation obviously relies on legacy syscalls.
			We can either try to run our deterministic user library alongside this
			legacy library, or write our own entirely from scratch.
	\end{itemize}
\fi

\subsection{High level approach}

To address concerns about Linux's more flexible process model, we present a
\emph{hybrid process model}. A Linux process invokes a syscall to become a
\emph{master} process, akin to Determinator's lone root process. This master
process has full access to the legacy Linux kernel API, with some minor
restrictions noted below. Master processes then create \emph{deterministic}
children. We call this master process and its entire subtree a
\emph{deterministic process group}. Within this process group, processes abide
by Determinator's nesting rules (e.g. children cannot outlive parents). A
deterministic process's death automatically triggers reaping that process's
subtree.

Legacy Linux applications run alongside deterministic applications with
absolutely no kernel restrictions. In some sense, each deterministic application
resembles an entire Determinator ``virtual machine'' of sorts.

We also add three new syscalls, {\tt dput()}, {\tt dget()}, and {\tt dret()}
and restrict deterministic processes to only use the new syscalls. These
syscalls function exactly as their Determinator counterparts, excepting cluster
support, the copy (grand)child subtree option, and instruction count limits.
By restricting deterministic processes to these three syscalls, we can nearly
remove all sources of nondeterminism; we only have to modify the kernel to
ignore all signals sent to deterministic processes, and thus we have effectively
blocked all sources of nondeterminism.

\iffalse
-- where to put root process limits (no KSM, no memadvise for huge pages)
We limit root processes in certain ways
\fi

\iffalse
	\item High level approach
	\begin{itemize}
		\item Hybrid process model: a Linux process becomes a root (akin to the
			lone Determinator root process) process via syscall. This process has
			(nearly) full access to Linux's syscall API. Restrictions explained
			later. Children of root, called deterministic, abide by Determinator's
			strict process hierarchy. Deterministic processes can only communicate
			with their parent and children using the three syscall API. All other
			Linux processes shall be called legacy; these processes can have no
			interaction with deterministic processes (signals, etc) except explicitly
			through the root process.
		\item Deterministic processes can only call three syscalls: dput(), dget(),
			and dret(); this restriction is nearly enough to block all
			non-deterministic inputs. All user generated signals to deterministic
			processes are ignored.
		\item Implement the three syscalls as described by Aviram (at a high level).
\fi

At the expense of predictability, but without harming determinism, master
processes in deterministic Linux are not restricted to blocking {\tt dput()}
and {\tt dget()}. A master can invoke these syscalls with a special flag to
poll whether or not the child process has reached a synchronization point yet.
We have to chosen to allow this, since some parallel applications fail to
exploit optimal concurrency in Determinator's design, like running a
{\tt make -j2}~\cite{Aviram10}. Applications that use the non-blocking syscalls
can still log schedule sequences to reproduce program output in a deterministic
fashion.

We also allow signals to reach master processes, and master processes can
specify a set of signals that can interrupt a blocking {\tt dput()} or
{\tt dget()}. Allowing signals for master processes again introduces
nondeterminism, but we note that it is explicit and controllable. We also note 
the usefulness of signals: terminal operators can send a {\tt SIGINT} to kill
an application immediately.

Once this kernel work is done, we begin work on a C user library. We won't use
the standard C library with deterministic processes, since many library calls
invoke disallowed legacy syscalls. The common use case of multithreaded
applications is to {\tt fork()} a child with a copy of the parent's virtual
memory image, thus giving the child access to the same library API as the
parent. This is undesirable for our system, since this would automatically let
deterministic children use the standard C library. To avoid this and namespace
problems (we want deterministic processes to use familiar function names like
{\tt printf()}), we require master and deterministic processes must use our new
deterministic library. Unfortunately, many functions must be rewritten (e.g.
{\tt sprintf()}, {\tt strlen()}).

To increase the usefulness of the system, we provide an in memory file system
just as Determinator does. Whereas Determinator's file system used fixed file
size~\cite{Aviram10cloud}, our file system design is similar to that of the BSD 
Fast File System~\cite{mckusick1984fast}. The file system is divided into
4096-byte \emph{blocks}. The first block is a \emph{superblock} containing
metadata about the file system. A region of fixed size following the superblock
is reserved for \emph{inodes} and a bitmap for managing free blocks. The rest of
the blocks are data blocks. In addition to direct block pointers, inodes support
a singly and doubly indirect block of data block pointers. Directories are files
containing a list of files within the directory.

We also note that since master processes have access to the system file system,
our user library can save the in memory file system to permanent storage if so
desired. Thus, our file system improves upon that of Determinator by supporting
hard linking, supporting larger file sizes, and being more flexible in managing
underlying resources (inodes, blocks).

\iffalse
		\item Once kernel work is finished, write user library to wrap around the
			three syscalls. Provide familiar fork(), wait(), and exec(), among others.
		\item The user library will be built from scratch: absolutely no libc. This
			gives us freedom in designing the in memory file system and implementations 
			of malloc(), etc. This user library should still behave as closely to the
			familiar C standard API (malloc, read, printf...).
		\item Write in memory file system with similar high level functionality as 
			that in Determinator. This file system also keeps track of standard input
			and output.
	\end{itemize}
\end{itemize}

\fi

\endinput

