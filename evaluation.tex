
\section{Evaluation}

This section evaluates deterministic Linux by running compute-bound
applications using deterministic Linux and nondeterministic pthreads.
We conclude by considering a case study demonstrating the qualitative debugging
benefits of determinism.

\subsection{Empirical experiments}

Aviram et al. already demonstrated course-grained applications in Determinator
performed comparable to nondeterministic Linux equivalents, but fine-grained
applications did not scale nearly as well, incurring high performance costs
owing to memory synchronization costs.


\begin{table*}[th]
\center{
\scalebox{.8}{
\begin{tabular}{|l| l l l l | l l l l |}
\hline
 & \multicolumn{4}{c|}{\textbf{lu}} & \multicolumn{4}{c|}{\textbf{matmult}} \\
Dimension & $N=1$ & $N=2$ & $N=4$ & $N=8$ & $N=1$ & $N=2$ & $N=4$ & $N=8$ \\
\hline
$16\times16$ & 13.1 (41.5\%) & 45.0 (46.7\%) & 46.3 (45.8\%) & 30.9 (31.5\%) &
  9.6 (48.2\%) & 21.8 (45.0\%) & 37.8 (45.2\%) & 16.0 (26.5\%) \\
$32\times32$ & 8.5 (34.0\%) & 37.3 (45.5\%) & 45.9 (46.1\%) & 29.1 (31.1\%) &
  3.3 (37.7\%) & 13.4 (42.0\%) & 21.1 (42.2\%) & 17.5 (24.1\%) \\
$64\times64$ & 2.6 (19.5\%) & 20.6 (41.6\%) & 42.1 (44.2\%) & 32.1 (30.9\%) &
  1.3 (13.4\%) & 3.8 (26.3\%) & 7.8 (34.0\%) & 13.2 (25.8\%) \\
$128\times128$ & 1.4 (2.3\%) & 6.0 (32.8\%) & 22.4 (39.0\%) & 30.8 (31.0\%) &
  1.0 (0.3\%) & 1.9 (1.7\%) & 4.5 (13.3\%) & 6.7 (18.2\%) \\
$256\times256$ & 1.1 (0.5\%) & 2.1 (11.0\%) & 7.0 (25.9\%) & 18.8 (31.1\%) &
  1.0 (0.0\%) & 1.2 (1.0\%) & 1.8 (1.6\%) & 2.3 (5.0\%) \\
$512\times512$ & 1.0 (0.1\%) & 1.2 (1.4\%) & 2.3 (9.4\%) & 5.9 (19.4\%) & 1.0
  (0.0\%) & 1.0 (0.5\%) & 1.1 (0.9\%) & 1.5 (3.0\%) \\
$1024\times1024$ & 1.4 (0.0\%) & 1.0 (0.3\%) & 1.2 (1.5\%) & 1.9 (8.4\%) & 1.0
  (0.0\%) & 0.9 (0.0\%) & 1.0 (0.1\%) & 1.2 (0.2\%) \\
$2048\times2048$ & 1.4 (0.0\%) & 1.0 (0.0\%) & 1.0 (0.1\%) & 1.1 (1.0\%) & - & - & - & - \\
\hline
\end{tabular}}}
\caption{Deterministic overhead for \emph{lu} and \emph{matmult}.
Overhead is deterministic run time divided by pthread run time.
The numbers in parentheses indicate time spent
in the kernel performing a virtual memory merge as a percentage of overall
runtime.}
\label{tab:matover}
\end{table*}

\endinput



\begin{table}[t]
\center{
\scalebox{.8}{
\begin{tabular}{|l| l l l l |}
\hline
 & \multicolumn{4}{c|}{\textbf{qsort}} \\
Input size & $N=1$ & $N=2$ & $N=4$ & $N=8$ \\
\hline
1K & 27.5 & 39.2 & 24.1 & 17.4 \\
4K & 14.1 & 46.8 & 37.9 & 20.8 \\
8K & 8.6 & 28.7 & 31.4 & 23.1 \\
10K & 7.4 & 21.2 & 30.6 & 26.6 \\
40K & 2.5 & 8.4 & 13.4 & 16.8 \\
80K & 1.7 & 4.3 & 8.4 & 10.6 \\
100K & 1.6 & 3.1 & 7.6 & 11.1 \\
400K & 1.2 & 1.8 & 2.2 & 2.7 \\
800K & 1.1 & 1.4 & 1.6 & 1.6 \\
1M & 1.1 & 1.4 & 1.5 & 1.8 \\
4M & 1.1 & 1.2 & 1.3 & 1.4 \\
8M & 1.0 & 1.1 & 1.2 & 1.4 \\
10M & 1.0 & 1.0 & 1.1 & 1.3 \\
40M & 1.0 & 1.1 & 1.2 & 1.3 \\
80M & 1.0 & 1.1 & 1.2 & 1.3 \\
\hline
\end{tabular}}}
\caption{Deterministic overhead for \emph{qsort}.}
\label{tab:qover}
\end{table}

\endinput



The primary goal of our evaluations is to determine if Linux can efficiently run
applications using Determinator's programming model. Since the deterministic
Linux kernel reuses a lot of existing code, we expect reasonable
performance compared to the equivalent nondeterministic application. However,
applications that heavily rely on memory-intensive kernel operations like
Snapshot/Merge might incur performance penalties.

\subsubsection{Methodology}
We evaluated four benchmark programs. \emph{md5} searches for a string whose
md5 hash yields a particular hash value (i.e. a brute force password cracker).
\emph{md5} creates $N+1$ threads to perform parallel work where $N$ is the
number of processors available. The \emph{qsort} benchmark recursively sorts an
integer array by forking children threads going $log_2(2\cdot N)$ levels deep,
after which the recursive computation is carried out sequentially.
\emph{matmult} multiplies two square matrices by dividing the input into
64 blocks and forking off threads to multiply the individual blocks. \emph{lu}
performs LU decomposition of a square matrix by breaking the input into 256
blocks and creating a thread for each block. Our LU algorithm breaks the blocks
up into discontiguous blocks of memory (i.e. we represent our matrix in
row-major order).

All benchmarks are designed so the deterministic and pthread versions operate on
identical inputs generated from a pseudorandom number generator (\emph{matult},
\emph{lu}, \emph{qsort}) or "hardcoded" input (\emph{md5}).
For a given input size, \emph{qsort} runs three tests on different inputs and
averages the run times. Both matrix benchmarks run ten tests on different inputs
and average the results. \emph{md5} searches for eleven fixed ASCII strings
whose values are distributed randomly over a fixed alphabet. The run times for
all eleven runs are averaged to a single value.

The \emph{matmult}, \emph{qsort}, and \emph{md5} benchmarks are modified
versions of benchmarks found in Determinator's source on
GitHub.~\footnote{\url{https://github.com/bford/Determinator}}
The \emph{lu} benchmark was written from scratch using an algorithm describe
on the Internet~\cite{lualg}. We tested on a 2 socket $\times$ 4 core 2.33 GHz
Intel Xeon PC running Arch Linux with 8GB of RAM. All benchmarks use the same
modified \mbox{x86\_64} Linux 3.0 kernel.

The deterministic benchmarks create threads by issuing a {\tt dput()} specifying
the {\tt Snap} option. We join thread results by performing a {\tt Merge} on the
program's entire static data segment (initialized and uninitialized). Thus,
deterministic applications incur the performance penalty of creating two copies
of the parent thread's page tables and associated kernel data structures
as well as scanning the page tables and potentially the pages
themselves of the child's entire static data. The pthread programs incur no
such penalties, since \mbox{{\tt pthread\_create()}} does not make copies of
page tables and \mbox{{\tt pthread\_join()}} does the same work as {\tt dget()}
without the {\tt Merge}.

\subsubsection{Results}
Table \ref{tab:matover} shows deterministic overhead for the \emph{lu} and
\emph{matmult} benchmarks. Small input sizes exhibit unacceptable overheads.
Inputs smaller than $1024\times 1024$ spend anywhere from $12.6\%$ to $58.2\%$
of their run time in the kernel doing a memory merge.
Small inputs in general have a hard time seeing parallel benefits: the pthread
versions of both benchmarks did not see parallel speedup benefits until the
input matrix reached at least $64\times 64$. 
It isn't until large input sizes (at least $1024\times 1024$) when we begin to
see more "acceptable" overheads of at most $2.1\times$ for $N>1$.
\emph{matmult} 

The embarrassingly parallel \emph{md5} benchmark showed very little overhead; it
had overhead of $1.17\times$ for $N=1$ and $1.06\times$ for $N=8$. This
can likely be attributed to the little amount of information transfered back and
forth between threads (a single boolean and the matching string when it is
found). The \emph{qsort} benchmark shows unacceptable overhead for small inputs,
but once the input array size becomes at least 800K, overheads stay under
$2\times$ (Table \ref{tab:qover}).

\subsection{Finding bugs deterministically}

To demonstrate one of the key benefits of deterministic (Section
\ref{sec:det-motiv}), we consider a Gaussian elimination program written using
the deterministic API and pthreads. Figures \ref{fig:pgauss} and
\ref{fig:dgauss} show nondeterministic and deterministic versions of Gaussian
elimination. There is a crucial synchronization bug, however: both algorithms
create \mbox{\tt nrows - k} worker threads but only join
\mbox{\tt nrows - k - 1} threads. We purposefully inserted this bug, but the
reader can imagine a programmer making this typo by mistake.

We ran both versions multiple times and examined the resulting matrix.
The deterministic program \emph{always}
produced the wrong answer. {\tt djoin()} merges changes back into the main
thread for all but the last worker thread. When the last worker thread
finishes, its changes are private and never seen by the main thread.

On the other hand, the pthreads program executes nondeterministically. We
observed three different output matrices, one of which was the correct result.
If the final worker thread finishes before we observe the final result, the
output will be correct. If the final worker thread does not finish by the time
we examine the output, we get an intermediate result. The output is
nondeterministic, owing to a race condition, since all threads share memory.

\begin{subfigures}
\begin{figure}
{\tt \footnotesize
\begin{verbatim}
pthread_t thread[MAXTHREADS];
struct thread_data data[MAXTHREADS];
void pthread_reduce(void) {
   for (i = 1; k <= nrows - 1; ++k) {
      for (i = k + 1; i <= nrows; ++i) {
         data[i] = /* Setup worker. */;
         pthread_create(&thread[i], NULL, worker,
            &data[i]);
      }

      /* Bug! Should be i <= nrows */
      for (i = k + 1; i < nrows; ++i)
         pthread_join(thread[i], NULL);
   }
}
\end{verbatim}
}
\caption{pthread Gaussian elimination.}
\label{fig:pgauss}
\end{figure}
\begin{figure}
{\tt \footnotesize
\begin{verbatim}
/* Forks a deterministic child. Returns 0 into the
 * child and 1 into the parent. */
int dfork(pid_t childid);
/* Merges a child's changes into the parent after
 * the child issues a dret(). */
void djoin(pid_t childid);

void det_reduce(void) {
   for (i = 1; k <= nrows - 1; ++k) {
      for (i = k + 1; i <= nrows; ++i) {
         data[i] = /* Setup worker. */;
         if (!dfork(i)) { worker(&data[i]); dret(); }
      }

      /* Bug! Should be i <= nrows */
      for (i = k + 1; i < nrows; ++i)
         djoin(i);
   }
}
\end{verbatim}
}
\caption{Deterministic Gaussian elimination.}
\label{fig:dgauss}
\end{figure}
\end{subfigures}

\endinput

