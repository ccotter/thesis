
\section{Limitations and future work}

[not finished]

I'll discuss Linux specific limitations like deterministic processes not using
the full memory subsystem feature set (e.g. huge pages). File system merging
is slow: because data blocks live in arbitrary regions throughout virtual
memory, reconciliation involves chasing data block pointers and invoking
dput/dget many times (linear in file size). This could be alleviated by
adding kernel support for an in-memory file system.

We don't support instruction limits, so malicious or buggy children could
execute forever, blocking parents indefinitely.

Implementation is not optimized. Example: killing a child process is
surprisingly slow (I'd like to find out why, since this should be a very quick
nonblocking operation).

\subsection{Future work}

Recent efforts by Determinator's authors have extended the OS to support
arbitrary process interactions (not just hierarchical). Determinator and
deterministic Linux focused on compute-bound applications using fork-join.
What about I/O intensive apps or other parallel paradigms?

\endinput

