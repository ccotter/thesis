
\section{Limitations and future work}


\begin{table}[t]
\center{
\scalebox{.84}{
\begin{tabular}{|l | r|}
\hline
Abstraction & Occurrence \\
\hline
Fork/Join & 17.9\% \\
Barrier & 14.8\% \\
Work Sharing Constructs & 32.8\% \\
%\multirow{2}{150px}{Deterministic through nondeterministic primitives} & \\
% & 26.7 \% \\
Deterministic through nondeterministic primitives & 26.7\% \\
Nondeterministic & 8.4\% \\
\hline
\end{tabular}}}
\caption{Synchronization abstractions used in the SPLASH, NPB-OMP, and PARSEC
benchmark programs.}
\label{tab:synctypes}
\end{table}

\endinput



While our user library supports the fork/join construct, there are other
parallel abstractions we could support. The barrier abstraction
blocks thread execution until all threads in in some program defined group
reach a certain point, upon which all threads in the group proceed to the next
barrier. The work sharing construct, used in the OpenMP parallel API, divides
work among a ``team'' of threads and blocks until all threads finish work. Our
user library support neither of these.

In his PhD thesis, Aviram classified the types of synchronization abstractions
used in the SPLASH, NBP-OMP, and PARSEC benchmark
programs~\cite{aviram2011deterministic}, and the results are summarized in Table
\ref{tab:synctypes}. The first three rows shows uses of naturally deterministic
library constructs; these account for 65.5\% of the total. Uses of
nondeterministic primitives to enforce deterministic behavior by the program
itself are shown in row four. The last row shows
uses nondeterministic abstractions like mutex locks and condition variables.
Future versions of our user library could implement barriers and the work sharing
parallel abstractions,
building upon the {\tt dput()}/{\tt dget()}/{\tt dret()} syscall primitives.
Doing so would appear to extend the types of parallel
programs our deterministic Linux could support.\footnote{Assuming the three
benchmark suites reflect real world parallel programs, as Aviram also
assumes~\cite{aviram2011deterministic}.}

Efficiency and overhead are important considerations in the attractiveness of
our system. As discussed in Section \ref{sec:finetune}, merging on a huge
region of memory is costly, even if we know at most a page or two of memory
could possibly have changed. We could enrich our library API to, for example,
narrow down the region that must be merged. If our kernel implementation allowed
deterministic processes to use huge pages, we could cut the number of page
table entries that must be examined by a factor of 500-1000.\footnote{Depending
on the huge page size (2MB or 4MB).}

Since the kernel API provides no direct file system support, our library uses
an in-memory file system. Our choice to use a Unix-like implementation may not
be well suited for reconciliation in user-space. If a file's data blocks do not
reside contiguously in memory, copying a file from a child into the parent may
require a {\tt dget()} call for each data block. Instead, we could move the
file system reconciliation logic into the kernel and extend the syscall API. A
more radical step would be to implement the entire file system itself in the
kernel, perhaps as part of Linux's Virtual File System API, the generic
interface to all file systems in Linux~\cite{vfs}.

Our deterministic Linux intentionally did not implement all of Determinator's
features. Determinator's {\tt Put} can limit the number of instructions a child
process can execute before an implicit {\tt Ret} is generated. Since
deterministic Linux does not support this, malicious or buggy children can run
in an infinite loop, blocking the parent indefinitely. The instruction count
feature is also crucial in emulating legacy thread APIs. Section 4.5 in Aviram
et al.'s Determinator paper describes how to implement a deterministic version
of the pthreads API~\cite{Aviram10}; a deterministic scheduler thread uses the
instruction count limit to ensure a deterministic (though arbitrary) schedule.

\endinput

