
\begin{abstract}
This thesis is about designing and implementing a deterministic programming
model in Linux based on Aviram et al.'s prior work, Determinator. Written from
the ground up, the Determinator operating system enforces deterministic process
execution by
removing implicit sources of nondeterminism (e.g. data races) and enforcing
strict synchronization rules that permit only ``naturally deterministic''
scheduling abstractions. Guided by Determinator's principles, we modify
Linux's process model and kernel interface to enforce determinism on user
programs, while maintaining backwards compatibility for legacy Linux
applications. We implement a basic user library for writing deterministic
applications and extend Determinator's in-memory file system by adding new
features and giving it persistence. Evaluations of compute-bound deterministic
applications against nondeterministic equivalents reveal unacceptable
overheads for small inputs; for large inputs, the overhead drops to less
than $2\times$ and the benchmarks begin to scale reasonably well.

\end{abstract}

\section{Introduction}
As processors move from single to multiple cores, more and more applications are
written parallel. Today, the dominant parallel programming model is
nondeterministic. In this model, threads typically share an entire address
space, file system, and other globally visible system managed resources like
process IDs. Operating systems schedule threads arbitrarily, and
lock abstractions are neither deterministic nor predictable. This model is
popular, because threads can operate on shared data ``in-place'' instead of
having to pack and unpack data structures~\cite{Aviram10}. Unfortunately, this
model is error prone and has many drawbacks~\cite{artho2003high,lee2006problem,
lu2008learning}. Programmers must eliminate data races introduced by
nondeterminism. Without repeatability, debugging, testing, and ensuring
software quality assurance become difficult.

Existing attempts to provide determinism require
special hardware~\cite{Devietti09,devietti2011rcdc}, custom programming
languages~\cite{bocchino2009type}, or specialized
compilers~\cite{bergan2010coredet}. Record-and-replay
systems~\cite{leblanc1987debugging,montesinos2008delorean} incur high
overhead and do not offer any insight into the inherently nondeterministic
behavior of a program. Systems that rely on a deterministic
scheduler residing in user space are often arbitrary and
unpredictable~\cite{olszewski2009kendo}; buggy or malicious application code
can compromise the scheduler~\cite{Aviram10,cui2010stable,bergan2010coredet}. 
Some relaxed deterministic models only enforce determinism on synchronization
and permit unsynchronized low-level memory accesses~\cite{olszewski2009kendo},
thus leading to nondeterministic traces in programs that do not properly protect
critical sections.

To overcome the challenges of nondeterminism, Aviram et al. presented a
deterministic operating system called Determinator~\cite{Aviram10}.
Programmers write parallel applications using a novel parallel programming model
that is ``naturally and pervasively deterministic'' and in fact predictable.
By enforcing determinism at the system level, Determinator run programs written
in general-purpose languages on conventional hardware.
Determinator builds a high-level library that provides familiar parallel
abstractions and an in-memory file system.
Evaluations of Determinator against Linux show that such a model can be
implemented to run coarse-grained parallel applications efficiently with little
overhead, but fine-grained parallel applications have unacceptable overhead.

\iffalse
# Determinator
* Aviram et al. presented a deterministic operating system called Determinator.
* Programs are written using a novel parallel programming model that is
  "naturally and pervasively deterministic" and in fact is predictable.
* Through a three syscall approach, the Determinator microkernel can run
  programs written in general-purpose languages like C on conventional hardware.
* Determinator also contributed a high level library with familiar abstractions
  and an in-memory file system.
* Evaluations of Determinator against Linux show that such a model can be
  implemented to run coarse-grained parallel application efficiently with little
  overhead, but fine-grained parallel applications have unacceptable overhead.

  [MW: above bullets are good. they're your prose, not Aviram et al.'s, right?]

# The state of parallel programming and nondeterminism
* As processes move from single to multiple cores, more and more applications
  are written parallel.
* Today, the dominant parallel programming model is nondeterministic. In this
  model:
* Threads share address space, file system, and other globally visible
  resources and
* The OS is free to schedule threads arbitrarily and lock abstractions are not
  deterministic or predictable.
* This model is popular despite drawbacks:
* Data races and lock abstractions introduce bugs and deadlock,
* Programmers spend a lot of time eliminating nondeterminism (data races) using
  unpredictable synchronization primitives,
* Programmers must worry about hardware side effects like ordering of committing
  memory operations,
* Debugging and quality assurance are difficult without repeatability.

For a deterministic Linux, we prefer strong determinism
that is predictable and enforced by the system. Our choice is
affirmed by Bochino et al., who claim that all parallel programming environments
must be ``deterministic by default''~\cite{bocchino2009parallel}.
\fi

This thesis is about adapting Determinator's operating system design and
programming model to Linux. We make the following contributions:
\begin{myitemize4}
    \item A presentation of a deterministic Linux kernel heavily based on
    the Determinator kernel. This is the first known adaptation of
    Determinator's kernel design in a widely deployed operating system.
    The bulk of the work in this area is implementation of an existing model.
    Additionally, we identify and eliminate Linux-specific sources of
    nondeterminism and present the \emph{hybrid process model}, an
    enhancement of Determinator's \emph{nested workspace model}. Then end
    result is that we can run legacy nondeterministic applications alongside
    deterministic applications in Linux.
    \item A deterministic high-level user library for Linux applications.
    This library is motivated by Determinator's user library, and
    again the bulk of the work in this area is implementation.
    \item An improvement of Determinator's in-memory file system.
    Determinator's file system maps a fixed number of files to static locations
    in memory and does not provide persistence. Our file system is modeled on
    the BSD Fast File System~\cite{mckusick1984fast}; thus, we support features
    like hard-linking and dynamic allocation of data blocks. We also provide
    persistence by saving the in-memory file system to Linux's disk-backed file
    system.
    \item An evaluation of deterministic Linux against traditional
    nondeterministic parallel Linux. We ran compute-bound parallel benchmarks
    and found that small input sizes incur unacceptable overheads of up to
    $39.2\times$ for the smallest input size. However, the overheads drop to
    less than $1.3\times$ for the largest inputs when run on multiple
    processors. We also demonstrate a case study of the qualitative benefits of
    determinism by studying a buggy parallel Gaussian elimination program.
\end{myitemize4}

\iffalse

# Contributions
* We make the following contributions:
* The first known adaptation of Determinator's design to a widely deployed OS.
* A high-level user library for application developers, similar to
  Determinator's user library.
* An improvement of Determinator's in-memory file system based on BSD Fast
  file system.
* An evaluation of deterministic Linux against legacy nondeterministic Linux and
  a case study of deterministic execution's benefits.

\fi

Aviram et al. were motivated by meeting the ``software development, debugging,
and security challenges'' of writing future parallel
applications~\cite{Aviram10}. Determinator was a huge step towards this.
Unfortunately, Determinator has limited uptake outside the academic community.
Linux is a mature and widely deployed operating system available for desktops,
servers, mobile, and embedded systems. Adding determinism alongside
nondeterministic Linux will be a huge next step. If we are lucky, we might be
able to influence how future parallel applications are written.

\iffalse

# Why?
* Aviram et al. were motivated by meeting the "software development, debugging,
  and security challenges" for parallel applications of the future. Determinator
  was a step towards this.
* Determinator itself has limited uptake, since it was written from scratch in
  an academic environment.
* Linux is a mature and widely deployed operating system; available for
  desktops, servers, mobile, embedded.
* If we can add determinism alongside nondeterministic Linux, this will be a
  huge next step.

\fi

[preview rest of paper]

\endinput

# Benefits
* According to Bergan et al., determinism provides benefits in four main areas:
  debugging, testing, fault tolerance, and security.
* Repeatability makes debugging easier and can be exploited to provide fault
  tolerance via replication.
* Determinism eliminates the exponential blow up of scheduling sequences.
* Determinism eliminates covert timing channels that can be used to extract
  sensitive data from privileged threads.
* Predictability allows better testing methodologies.

(will elaborate on benefits in separate section)

[MW: state the benefits in one or two sentences (four is too many), and
say that you'll elaborate later.]


