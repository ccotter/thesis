% ORGANIZATION OF THE PREAMBLE
\usepackage{tabularx}           % more table control
% \usepackages
% definitions for editing marks
% document-specific crap

\usepackage[pdftex]{graphicx}   % allow graphics
\usepackage{subfigure}          % allow sub-figures
\usepackage{booktabs}           % pretty tables
\usepackage{dcolumn}            % align decimal points
\usepackage{multirow}           % for tables
%\usepackage{arydshln}           % allow dashed lines in tables
\usepackage{rotating}           % allow rotated text 
\usepackage{xspace}             % define commands that don't eat space
\usepackage{hyphenat}           % supplies \hyp{}, which tells tex that it can 
		                % hyphenate at an existing hyphen
\usepackage{enumerate}          % more control over enumerates 
\usepackage{dblfloatfix}        % tame those floats
\usepackage{lastpage}           % allow the last page to be referenced
\usepackage{pifont}             % dingbat fonts
\usepackage[normalem]{ulem}     % included for strike-out {\sout X}
\usepackage[table]{xcolor}      % color, for editing marks.
\usepackage{colortbl}           % colorful columns and rows
\usepackage{array}              % needed for 'b' argument in tabular preamble
\usepackage{tabularx}           % more table control
%\usepackage[noeka]{mathrmletter}% Eddie Kohler's mathrm

% pretty algorithms
%\usepackage[noend]{algpseudocode}
%\algrenewcomment[1]{\hfill// #1}%   % don't include end-block syntax

% AMS stuff, so we have no surprises when we use math
\usepackage{amsmath}
\usepackage{amssymb}
\usepackage{amsfonts}

% peanut gallery comments
% NOTE: Comment out no editing marks line in the main .ltx 
%
%\newcommand{\textred}[1]{\textcolor{red}{#1}}
\newcommand{\textred}[1]{\begingroup \color{red} #1\endgroup}
\ifx\noeditingmarks\undefined%
   \newcommand{\pgwrapper}[2]{\textred{#1: #2}}
   \newcommand{\pgwrapperb}[1]{\textbf{#1}}
\else%
   \newcommand{\pgwrapperb}[1]{}
   \newcommand{\pgwrapper}[2]{}
\fi%

\newcommand{\jbl}[1]{\pgwrapper{JBL}{#1}}
\newcommand{\mw}[1]{\pgwrapper{MW}{#1}}
% end peanut gallery comments

\ifx\noeditingmarks\undefined%
    \newcommand{\changebars}[2]{%
    [{\em \begingroup \color{magenta} #1 \endgroup}]
    [\begingroup \color{magenta} \sout{#2} \endgroup]}
\else%
    \newcommand{\changebars}[2]{#1}
\fi%

\newcommand{\goodcitationsize}{\fontsize{8}{9.6}\selectfont}
\newcommand{\annoyingsize}{\fontsize{7.5}{9}\selectfont}
\newcommand{\moreannoyingsize}{\fontsize{7.3}{8.8}\selectfont}
\newcommand{\acmgarbagesize}{\fontsize{7.5}{9}\selectfont}

% definitions for MP
\def\hn{\usefont{OT1}{phv}{mc}{n}\selectfont}
\def\hb{\usefont{OT1}{phv}{bc}{n}\selectfont}
\def\hv{\usefont{OT1}{phv}{m}{n}\selectfont}
\newcommand{\mpfont}{\hn\scriptsize}
\newcommand{\MPworker}[2]{{\color{#1}\vrule\vrule}{\marginpar{\color{#1}\mpfont%
                           #2}}}
\ifx\noeditingmarks\undefined%
    \newcommand{\MP}[1]{\MPworker{blue}{#1}}
    \newcommand{\MPjbl}[1]{\MPworker{red}{#1}}
\else%
   \newcommand{\MP}[1]{}
    \newcommand{\MPjbl}[1]{}
\fi%
\setlength{\marginparwidth}{17mm}
\setlength{\marginparsep}{0.35mm}

\newcommand\rmv[1]{}

\def\t{\textit}
\def\as{\leftarrow}

\newcommand{\citepaxosusers}{\cite{burrows06chubby,john08consensus,%
                                   chandra07paxos,zookeeper,lee96petal,%
                                   stribling09flexible,boxwood,bolosky11paxos}}

\newcommand{\circledone}{\ding{192}\xspace}
\newcommand{\circledtwo}{\ding{193}\xspace}
\newcommand{\circledthree}{\ding{194}\xspace}
\newcommand{\circledfour}{\ding{195}\xspace}
\newcommand{\circledfive}{\ding{196}\xspace}
\newcommand{\filledone}{\ding{202}\xspace}
\newcommand{\filledtwo}{\ding{203}\xspace}
\newcommand{\filledthree}{\ding{204}\xspace}
\newcommand{\filledfour}{\ding{205}\xspace}
\newcommand{\filledfive}{\ding{206}\xspace}

\newcommand{\cpu}{\textsc{cpu}\xspace}
\newcommand{\usec}{$\mu\mathrm{s}$}

\newcommand{\so}{\mbox{F-alive}\xspace}
\newcommand{\si}{\mbox{F-crash}\xspace}
\newcommand{\sii}{\mbox{F-cong}\xspace}
\newcommand{\siii}{\mbox{F-route}\xspace}
\newcommand{\siv}{\mbox{F-part}\xspace}

\def\compactify{\itemsep0in \topsep2pt \parsep=0.00in \partopsep=0pt \leftmargin4em}
\let\latexusecounter=\usecounter

\newenvironment{CompactItemize}%
  {\def\usecounter{\compactify\latexusecounter}%
   \begin{itemize}}%
  {\end{itemize}\let\usecounter=\latexusecounter}%
\newenvironment{CompactEnumerate}%
  {\def\usecounter{\compactify\latexusecounter}%
   \begin{enumerate}}%
  {\end{enumerate}\let\usecounter=\latexusecounter}

\newenvironment{myenumerate}%
  {\def\usecounter{\compactify\latexusecounter}%
   \begin{enumerate}}%
  {\end{enumerate}\let\usecounter=\latexusecounter}

\newenvironment{myitemize}%
  {\begin{list}%
    {\labelitemi}{\itemsep0in \topsep2pt \parsep0.00in \partopsep=0pt \leftmargin\parindent}}%
  {\end{list}}

\newenvironment{myenumerate2}%
  {\def\usecounter{\itemsep=0ex \topsep1ex \parsep=1ex \partopsep=0pt \leftmargin\parindent\latexusecounter}%
   \begin{enumerate}}%
  {\end{enumerate}\let\usecounter=\latexusecounter}

\newenvironment{myenumerate3}%
  {\def\usecounter{\itemsep0pt \topsep0pt \parsep=0ex \partopsep=0pt \leftmargin\parindent\latexusecounter}%
   \begin{enumerate}}%
  {\end{enumerate}\let\usecounter=\latexusecounter}

\newenvironment{myitemize2}%
  {\begin{list}%
    {\labelitemi}{\itemsep6pt \topsep6pt \parsep0.00in \partopsep=0pt \leftmargin\parindent}}%
  {\end{list}}

\newenvironment{myitemize3}%
  {\begin{list}{\labelitemi}{\itemsep3pt \topsep3pt \parsep0.00in
  \partopsep=0pt \leftmargin\parindent}}%
  {\end{list}}

\newenvironment{myitemize4}%
  {\begin{list}{\labelitemi}{\itemsep0in \topsep2pt \parsep0.00in \partopsep=0pt \leftmargin\parindent}}%
  {\end{list}}

%\def\emparagraph#1{\vspace{0.7mm}\noindent{\bf #1}}
\def\emparagraph#1{\paragraph{#1}}

\newcommand{\astskip}{\smallskip\noindent\parbox{\linewidth}
			{\center*\hspace{2.5em}*\hspace{2.5em}*\medskip\smallskip}}

\def\discretionaryslash{\discretionary{/}{}{/}}
{\catcode`\/\active%
\gdef\URLprepare{\catcode`\/\active\let/\discretionaryslash
        \def~{\char`\~}}}%
\def\URL{\bgroup\URLprepare\realURL}%
\def\realURL#1{\tt #1\egroup}%

\newcommand{\tmidrule}{\midrule[0.02em]}
