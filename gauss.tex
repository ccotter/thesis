
\begin{subfigures}
\begin{figure}[t]
{\tt \footnotesize
\begin{verbatim}
pthread_t thread[MAXTHREADS];
struct thread_data data[MAXTHREADS];
void pthread_reduce(void) {
   for (i = 1; k <= nrows - 1; ++k) {
      for (i = k + 1; i <= nrows; ++i) {
         data[i] = /* Setup worker. */;
         pthread_create(&thread[i], NULL, worker,
            &data[i]);
      }

      /* Bug! Should be i <= nrows */
      for (i = k + 1; i < nrows; ++i)
         pthread_join(thread[i], NULL);
   }
}
\end{verbatim}
}
\caption{pthread Gaussian elimination.}
\label{fig:pgauss}
\end{figure}
\begin{figure}[t]
{\tt \footnotesize
\begin{verbatim}
/* Forks a deterministic child. Returns 0 into the
 * child and 1 into the parent. */
int dfork(pid_t childid);
/* Merges a child's changes into the parent after
 * the child issues a dret(). */
void djoin(pid_t childid);

void det_reduce(void) {
   for (i = 1; k <= nrows - 1; ++k) {
      for (i = k + 1; i <= nrows; ++i) {
         data[i] = /* Setup worker. */;
         if (!dfork(i)) { worker(&data[i]); dret(); }
      }

      /* Bug! Should be i <= nrows */
      for (i = k + 1; i < nrows; ++i)
         djoin(i);
   }
}
\end{verbatim}
}
\caption{Deterministic Gaussian elimination.}
\label{fig:dgauss}
\end{figure}
\end{subfigures}

\endinput

