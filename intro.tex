
\section{Introduction}
\label{s:intro}

This thesis describes adding kernel enforced deterministic program execution to
Linux. We describe the challenges, design, implementation, and evaluation of
a deterministic Linux.
In our solution, at a high level, programs enter a \emph{deterministic}
mode where the kernel provides a very restricted subset of syscalls designed to
enforce determinism.

\iffalse
A program is deterministic if for a fixed input, the program always returns the
same output. This is desirable, because it simplifies testing and debugging of
ever more ubiquitous parallel programs. Linux's parallel programming model is
inherently nondeterministic, but having the option to run deterministically
would be a great feature as computing moves towards using multiple cores.
\fi

We are motivated by the difficulties of parallel programming in the presence
of nondeterminism. The multicore revolution has encouraged parallel programs
over sequential. The inherent nondeterminism in the conventional threading
model poses a threat to the quality and correctness of future
applications~\cite{lee2006problem}. Bocchino et al. argue that parallel programs
must be programmed with a model that is ``deterministic by
default''~\cite{bocchino2009parallel}.

Data races and other nondeterministic inputs force programmers to use
difficult to reason about synchronization primitives like semaphores and
condition variables. Misuse of these primitives can lead to buggy code and
deadlock. Even correct use cannot guarantee deterministic execution:
conventional synchronization primitives are not predictable~\cite{Aviram10}.
Determinism is so highly sought, because it overcomes the challenges of
nondeterminism. According to Bergan et al., determinism provides benefits in four
main areas: debugging, fault tolerance, testing, and security~\cite{Bergan11}.

\iffalse
We demonstrate the debugging benefits by discussing a
buggy LU factorization program: whereas a
critical bug manifests only \emph{sometimes} running with nondeterministic
pthreads, deterministic Linux exhibits the critical bug every time the program
runs. Because of this, finding and eliminating the bug becomes much easier. We
elaborate on Bergan's four benefits in a subsequent discussion.
\fi

The research presented in this thesis is based on Determinator~\cite{Aviram10},
a deterministic operating system. We adapt Aviram et al.'s operating system
design to make Linux deterministic. The end result is that we are able to write
user programs that are indeed deterministic by default: user programs may not
execute nondeterministically, even by deliberate design. We choose to adapt
Determinator, since it requires no special hardware or specialized programming
languages; instead, Determinator enforces determinism through a microkernel
approach syscall interface. We are able to write programs in general purpose
languages and run them on non-specialized hardware.

This thesis makes the following contributions:

\begin{itemize}
    \item A presentation of a deterministic Linux kernel heavily based on that
    of the Determinator kernel. This is the first known adaptation of
    Determinator's kernel design in a \emph{real} operating system.
    \item A deterministic high level user library for use by application
    programmers. This library is motivated by Determinator's user library and
    the usefulness developing programs using an API similar to that of the
    standard C library.
    \item An improvement of Determinator's in memory file system. Our file
    system is modeled off of the BSD Fast File System~\cite{mckusick1984fast},
    and it provides persistence.
    \item We evaluate the performance of deterministic Linux against traditional
    nondeterministic parallel Linux. We also demonstrate a case study of the
    benefits of determinism.
\end{itemize}

\iffalse
This thesis also presents an accompanying
user level C library, akin to the library utilities discussed by Aviram et al.
This user library is intended to simplify writing user programs in C using
familiar high level abstractions such as fork-join. We also provide an in memory
file system, improving upon Arivam et al.'s in memory file system design.
\fi

Determinator contributed a novel programming model building off of existing
ideas like transactional memory~\cite{herlihy1993transactional} and distributed
shared memory (DSM)~\cite{amza1996treadmarks}. Written from scratch in an
academic setting, Determinator has limited uptake in the wider computing
community. Linux is a widely deployed open source operating system with a
mature, advanced feature set and countless programming libraries and
applications; this makes it a very attractive target for providing determinism.
If we are lucky, we might be able to influence how future parallel applications
are written.

Evaluations show that our deterministic Linux has performance comparable to
that of nondeterministic Linux using pthreads. Embarrassingly applications
written using deterministic abstractions have little overhead. We are optimistic
that the benefits of deterministic execution over the serious drawbacks of
programming with nondeterminism, coupled with additional kernel features and
improvements in library design will make deterministic Linux a popular choice
in the future of parallel application development.

\iffalse
The rest of this thesis discusses what makes developing with nondeterminism
difficult and why deterministic execution is desirable. We then present Aviram
et al.'s Determinator kernel design in detail, as well as the challenges
Determinator had to overcome. We then motivate adding determinism to a
\emph{real} operating system, Linux.

Next, we present our design goals for our work, centering our our desire to add
kernel enforced determinism to the Linux kernel. We discuss the challenges of
adding determinism to an inherently nondeterministic kernel, and then we present
a high level design of our system.

With that in mind, we go into the details of how we actually made Linux
deterministic; we discuss the specific implementation details of what we changed
in the Linux kernel, and we also describe building a user library based on the
new kernel API. We then evaluate the performance of applications running on
deterministic Linux and compare the results to the equivalent applications
using conventional nondeterministic Linux.

The section following this introduction gives background on Determinator and
further motivates making Linux deterministic. The next section will describe
design goals for this research project, followed by a discussion of challenges
in accomplishing the goals. Then, we will discuss the actual implementation
details of the kernel and user library work.

In later sections, we evaluate deterministic Linux against legacy Linux and
make some conclusions about Determinator's design in Linux compared to Linux's
traditional multithreaded runtime. We will also discuss limitations and related
work. Finally, we will begin to conclude by discussing the overall undergraduate
research experience.

Nondeterministic inputs, such as scheduling order and data
races, that should have no effect on determining a program's output no longer
affect a Linux programs 
only supports three syscalls
 execution is a useful aspect of a computer program. Determinism
means that for any fixed input, a program always returns the same output.
Multithreaded programs inherently introduce mechanisms that make deterministic
execution difficult, if not impossible. Deterministic execution is desirable,
because it has debugging and security implications and makes reasoning about
code correctness easier.

Multithreaded programming introduces many sources of non-determinism, such as
shared memory. Execution aspects, particularly scheduling of threads, of a
parallel program depends on arbitrary data races, and these data races can
manifest in unpredictable ways. Because these data races manifest unpreditably
and rely on particular the particular hardware on which the software runs, bugs
might not appear until a parallel progam is run on an end-user's machine making
debugging all the more difficult.

Conventional multi-threaded programming supports a ``share everything'' model.

All threads concurrently access and write to shared memory, and threads share a
common file system, threads are not required to explicitly synchronize or wait
for other threads, and the preemptive multi-threading, threads may be scheduled
arbitrarily.

The Determinator OS took a novel approach for providing deterministic program
execution: the OS enforces determinism on user-level programs through kernel
mechanisms. User programs do not have direct access to hardware resources and
are not permitted to share anything, including memory, without explicit
operating system support (via syscalls).

Determinator sought to provide a new programming model to counter the
conventional multi-threaded model. The strongest guarantee of the new model
is that Determinator is ``deterministic by default.'' In other words, user-level
programs cannot by any means, either intentional or unintentional, execute
non-deterministically. Non-deterministic inputs, such as time of day or access
to the shared file system, are made explicit. The kernel also requires threads
to explicitly synchronize. Through these mechanisms, the Determinator model
eliminates data races entirely.

The goal of this thesis is to discuss the incorporation of the core ideas of
Determinator into the Linux operating system. Next, this thesis will describe
how the Linux kernel was modified to support a deterministic process model.
Determinator also contributed a user-level programming library to support
familiar APIs, such as pthreads and file system access; this thesis also
presents an accompanying user library for Linux programs.

\fi
\endinput

