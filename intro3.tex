
\section{Introduction}
As processes move from single to multiple cores, more and more applications are
written parallel. Today, the dominant parallel programming model is
nondeterministic. In this model, threads typically share an entire address
space, file system, and other globally visible system managed resources like
process IDs. The operating system freely schedules threads arbitrarily, and
lock abstractions are not deterministic or predictable.

This model is popular,
because threads can operate on shared data ``in-place'' instead of having to
pack and unpack data structures~\cite{Aviram10}. Unfortunately, this model is
error prone and has many drawbacks. Programmers spend a lot of time eliminating
data races introduced by nondeterminism using unpredictable synchronization
primitives. Programmers must worry about hardware side effects like in what
order the processor commits memory write operations. Debugging and quality
assurance are difficult without repeatability.

\iffalse
# The state of parallel programming and nondeterminism
* As processes move from single to multiple cores, more and more applications
  are written parallel.
* Today, the dominant parallel programming model is nondeterministic. In this
  model:
* Threads share address space, file system, and other globally visible
  resources and
* The OS is free to schedule threads arbitrarily and lock abstractions are not
  deterministic or predictable.
* This model is popular despite drawbacks:
* Data races and lock abstractions introduce bugs and deadlock,
* Programmers spend a lot of time eliminating nondeterminism (data races) using
  unpredictable synchronization primitives,
* Programmers must worry about hardware side effects like ordering of committing
  memory operations,
* Debugging and quality assurance are difficult without repeatability.
\fi

In general, a program is a function of both implicit and explicit inputs.
We say that inputs to a program that are semantically relevant to the
application are explicit and otherwise implicit. Most implicit inputs are
random, arbitrary, and uncontrollable; examples are timing dependencies, quantum
size, and cache size. We say a program is \emph{deterministic} if it is
a function of only its explicit inputs. Even though a program's schedule
sequence may be deterministic, it may still be \emph{arbitrary}. *sdf*

\iffalse
# Introduce determinism
* A program is a function of its inputs, both implicit and explicit.
* We say that inputs that are semantically relevant to an application are
  explicit, and otherwise implicit.
* Most implicit inputs are random, arbitrary, and uncontrollable. Examples:
  timing dependencies, quantum size, cache size.
* A program is deterministic if it is a function of only explicit inputs.
[MW: the three bullets above are careful and logical but could be
expressed with a small number of words.]
* Strong determinism guarantees a deterministic order of all shared memory
  accesses; *all* programs run deterministically in such an environment.
* Weak determinism only guarantees a deterministic order of lock acquisitions;
  data races can still lead to a nondeterministic execution.
[MW: at this point in the flow, this point about weak determinism will
distract. you can talk about that later. just defined "determinism" in
the way that is most convenient and keep going.]
* Deterministic schedule abstractions might be arbitrary and unpredictable: the
  next thread to acquire a mutex lock could depend on hardware implementation or
  instruction counting.
* When considering a program as a relation in the mathematical sense,
  determinism guarantees the relation is a one-to-one function, but we may not
  know how to compute the function without running the program.
* If we can determine a program's output from program logic alone, we say the
  program is predictable; predictability is a stronger notion than determinism.
[MW: you _may_ wish to: (1) just define deterministic as you are currently
defining "predictable" (2) include a footnote saying, "Actually, this is
a stronger definition than is usual. See Section [relwork] (3) in the
related work section, clarify that your definition of determinism is
stronger.]

# Benefits
* According to Bergan et al., determinism provides benefits in four main areas:
  debugging, testing, fault tolerance, and security.
* Repeatability makes debugging easier and can be exploited to provide fault
  tolerance via replication.
* Determinism eliminates the exponential blow up of scheduling sequences.
* Determinism eliminates covert timing channels that can be used to extract
  sensitive data from privileged threads.
* Predictability allows better testing methodologies.

(will elaborate on benefits in separate section)

[MW: state the benefits in one or two sentences (four is too many), and
say that you'll elaborate later.]

# What kind of determinism do we want?
* Determinism can be provided by special hardware (custom instruction sets),
  specialized programming languages, or the operating system (or a
  combination).
* Environments like DPJ require special programming languages and have limited
  uptake; however, we want to write programs in general-purpose languages.
* Systems like DMP require special hardware support. Again, these systems have
  limited uptake; we want to write programs on widely available/popular CPUs.
* Kendo only enforces scheduling synchronization and is only weakly
  deterministic.
* We prefer strong determinism, because the burden of providing determinism
  falls on the system, not the programmer. Programs run in a strongly
  deterministic environment cannot possibly behave nondeterministically.
* We also want predictability; Kendo's uses a "wait for turn" approach where
  schedule sequences are arbitrary and depend on instruction counting.
* With all of the above aspects, we can write, run, and debug programs knowing
  that output is determined only by program logic.

[MW: above is a bit too much detail for this point in the flow. Why does
reader need to understand DPJ, DMP, etc.? If you can get to Determinator
sooner, or at least foreshadow it, that'd be better. ideally, you're
getting to Determinator in the 3rd or 4th paragraph of the intro (not
later). I think you can do that.]

# Determinator
* Aviram et al. presented a deterministic operating system called Determinator.
* Programs are written using a novel parallel programming model that is
  "naturally and pervasively deterministic" and in fact is predictable.
* Through a three syscall approach, the Determinator microkernel can run
  programs written in general-purpose languages like C on conventional hardware.
* Determinator also contributed a high level library with familiar abstractions
  and an in-memory file system.
* Evaluations of Determinator against Linux show that such a model can be
  implemented to run coarse-grained parallel application efficiently with little
  overhead, but fine-grained parallel applications have unacceptable overhead.

[MW: above bullets are good. they're your prose, not Aviram et al.'s,
right?]

# Introduce deterministic Linux
* This thesis is about adapting Determinator's operating system design and
  programming model to Linux.
* We implement Determinator's three syscalls inside the Linux kernel.
* We can run legacy nondeterministic programs alongside deterministic programs,
* We can write parallel applications using familiar abstractions, but legacy
  applications must still be rewritten.

# Why?
* Aviram et al. were motivated by meeting the "software development, debugging,
  and security challenges" for parallel applications of the future. Determinator
  was a step towards this.
* Determinator itself has limited uptake, since it was written from scratch in
  an academic environment.
* Linux is a mature and widely deployed operating system; available for
  desktops, servers, mobile, embedded.
* If we can add determinism alongside nondeterministic Linux, this will be a
  huge next step.

# Contributions
* We make the following contributions:
* The first known adaptation of Determinator's design to a widely deployed OS.
* A high-level user library for application developers, similar to
  Determinator's user library.
* An improvement of Determinator's in-memory file system based on BSD Fast
  file system.
* An evaluation of deterministic Linux against legacy nondeterministic Linux and
  a case study of deterministic execution's benefits.
\fi

\endinput


