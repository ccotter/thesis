
\begin{abstract}
This is Determinator in Linux. Hope you read it all.
\end{abstract}

\section{Introduction}
As processors move from single to multiple cores, more and more applications are
written parallel. Today, the dominant parallel programming model is
nondeterministic. In this model, threads typically share an entire address
space, file system, and other globally visible system managed resources like
process IDs. Operating systems schedule threads arbitrarily, and
lock abstractions are not deterministic or predictable.

This model is popular,
because threads can operate on shared data ``in-place'' instead of having to
pack and unpack data~\cite{Aviram10}. Unfortunately, this model is
error prone and has many drawbacks. Programmers spend a lot of time eliminating
data races introduced by nondeterminism by using unpredictable synchronization
primitives. Programmers must worry about hardware side effects like commit
ordering of memory write operations. Without repeatability, debugging and
ensuring software quality assurance become difficult.

To overcome the challenges of nondeterminism, Aviram et al. presented a
deterministic operating system called Determinator~\cite{Aviram10}.
Programs are written using a novel parallel programming mode that is ``naturally
and pervasively deterministic'' and in fact predictable. Through a three
syscall approach, the
Determinator microkernel can run programs written in general-purpose languages
like C on conventional hardware. Determinator also constributes a high-level
library with familiar abstractions and an in-memory file system. Evaluations
of Determinator against Linux show that such a model can be implemented to run
coarse-grained parallel applications efficiently with little overhead, but
fine-grained parallel applications have unacceptable overhead.

\iffalse

# Determinator
* Aviram et al. presented a deterministic operating system called Determinator.
* Programs are written using a novel parallel programming model that is
  "naturally and pervasively deterministic" and in fact is predictable.
* Through a three syscall approach, the Determinator microkernel can run
  programs written in general-purpose languages like C on conventional hardware.
* Determinator also contributed a high level library with familiar abstractions
  and an in-memory file system.
* Evaluations of Determinator against Linux show that such a model can be
  implemented to run coarse-grained parallel application efficiently with little
  overhead, but fine-grained parallel applications have unacceptable overhead.

[MW: above bullets are good. they're your prose, not Aviram et al.'s,
right?]

\fi

\iffalse
# The state of parallel programming and nondeterminism
* As processes move from single to multiple cores, more and more applications
  are written parallel.
* Today, the dominant parallel programming model is nondeterministic. In this
  model:
* Threads share address space, file system, and other globally visible
  resources and
* The OS is free to schedule threads arbitrarily and lock abstractions are not
  deterministic or predictable.
* This model is popular despite drawbacks:
* Data races and lock abstractions introduce bugs and deadlock,
* Programmers spend a lot of time eliminating nondeterminism (data races) using
  unpredictable synchronization primitives,
* Programmers must worry about hardware side effects like ordering of committing
  memory operations,
* Debugging and quality assurance are difficult without repeatability.
\fi

Whereas Determinator was written from scratch in an academic environment,
we would like a deterministic environment in a more widely deployed operating
system. The research presented in this thesis is based on applying
Determinator's model to Linux. Before we describe the work of this thesis, we
will present background information on determinism and argue why we chose to
base our work on Determinator.

In general, a program is a function of both implicit and explicit inputs.
We call semantically relevant inputs \emph{explicit} and otherwise
\emph{implicit}. Most implicit inputs are
random, arbitrary, and uncontrollable; examples are timing dependencies, quantum
size, and cache size~\cite{Bergan11}.~\footnote{The reader should note the
subtle distinction in classifying inputs as explicit or implicit and
deterministic or nondeterministic. Time of day is often an explicit
nondeterministic input, since it is semantically relevant to many applications.
For a given CPU, cache size is an implicit deterministic input.}
We say a program is \emph{deterministic}
if it is a function of only its explicit inputs.

Even though a program's schedule
sequence may be deterministic, it may still be \emph{arbitrary}. The next
thread to acquire a mutex lock could depend on hardware implementation or
instruction counting~\cite{olszewski2009kendo}. If we consider a program
as a relation in the
mathematical sense, determinism guarantees the relation is a one-to-one
function, but we may not know how to compute the function without actually
running the program. If we can determine a program's output from examining the
program code, we say the program is \emph{predictable}. For the rest of this
thesis, we modify our definition of determinism to have the property of
predictability and make the distinction explicit only when
required.~\footnote{The usual definition of determinism does not include the
stronger notion of predictability, but it will be useful for our purposes to
include it.}

\iffalse
# Introduce determinism
* A program is a function of its inputs, both implicit and explicit.
* We say that inputs that are semantically relevant to an application are
  explicit, and otherwise implicit.
* Most implicit inputs are random, arbitrary, and uncontrollable. Examples:
  timing dependencies, quantum size, cache size.
* A program is deterministic if it is a function of only explicit inputs.
[MW: the three bullets above are careful and logical but could be
expressed with a small number of words.]
* Strong determinism guarantees a deterministic order of all shared memory
  accesses; *all* programs run deterministically in such an environment.
* Weak determinism only guarantees a deterministic order of lock acquisitions;
  data races can still lead to a nondeterministic execution.
[MW: at this point in the flow, this point about weak determinism will
distract. you can talk about that later. just defined "determinism" in
the way that is most convenient and keep going.]
* Deterministic schedule abstractions might be arbitrary and unpredictable: the
  next thread to acquire a mutex lock could depend on hardware implementation or
  instruction counting.
* When considering a program as a relation in the mathematical sense,
  determinism guarantees the relation is a one-to-one function, but we may not
  know how to compute the function without running the program.
* If we can determine a program's output from program logic alone, we say the
  program is predictable; predictability is a stronger notion than determinism.
[MW: you _may_ wish to: (1) just define deterministic as you are currently
defining "predictable" (2) include a footnote saying, "Actually, this is
a stronger definition than is usual. See Section [relwork] (3) in the
related work section, clarify that your definition of determinism is
stronger.]

\fi

There are many factors to consider in designing a deterministic environment.
Some systems provide determinism through special
hardware, but these systems have limited uptake, We want to write programs
on widely available CPUs. Some environments provide determinism through
special programming languages, but again these have limited uptake. We want to
write programs in conventional general-purpose programming languages.
Enforcing deterministic scheduling through user space libraries has the drawback
that programming bugs or malicious code can interfere with the scheduler.
Instead, we want a system that cannot be compromised by misbehaved code.
With system enforced deterministic, the burden shifts from the programmer to the
system. Lastly, we consider the two variations of determinism: strong and
weak~\cite{olszewski2009kendo}. \emph{Strong} determinism
ensures a deterministic order of all memory accesses, and every program is
guaranteed to run deterministically. \emph{Weak} determinism ensures a
deterministic sequence of lock acquisitions, but the existence of data races
can lead to non deterministic program execution.

For a deterministic Linux, we prefer strong determinism
that is predictable and enforced by the system. Our choice is
affirmed by Bochino et al., who claim that all parallel programming environments
must be ``deterministic by default''~\cite{bocchino2009parallel}.

\iffalse

# What kind of determinism do we want?
* Determinism can be provided by special hardware (custom instruction sets),
  specialized programming languages, or the operating system (or a
  combination).
* Environments like DPJ require special programming languages and have limited
  uptake; however, we want to write programs in general-purpose languages.
* Systems like DMP require special hardware support. Again, these systems have
  limited uptake; we want to write programs on widely available/popular CPUs.
* Kendo only enforces scheduling synchronization and is only weakly
  deterministic.
* We prefer strong determinism, because the burden of providing determinism
  falls on the system, not the programmer. Programs run in a strongly
  deterministic environment cannot possibly behave nondeterministically.
* We also want predictability; Kendo's uses a "wait for turn" approach where
  schedule sequences are arbitrary and depend on instruction counting.
* With all of the above aspects, we can write, run, and debug programs knowing
  that output is determined only by program logic.

[MW: above is a bit too much detail for this point in the flow. Why does
reader need to understand DPJ, DMP, etc.? If you can get to Determinator
sooner, or at least foreshadow it, that'd be better. ideally, you're
getting to Determinator in the 3rd or 4th paragraph of the intro (not
later). I think you can do that.]

\fi

This thesis is about adapting Determinator's operating system design and
programming model to Linux. We implement Determinator's three syscall interface
inside the Linux kernel, then we write a high-level user library based on that
of Determinator. We can run legacy nondeterministic programs alongside
deterministic programs. We can write new parallel applications using
familiar abstractions. Legacy applications must be rewritten using the new
API.

\iffalse

# Introduce deterministic Linux
* This thesis is about adapting Determinator's operating system design and
  programming model to Linux.
* We implement Determinator's three syscalls inside the Linux kernel.
* We can run legacy nondeterministic programs alongside deterministic programs,
* We can write parallel applications using familiar abstractions, but legacy
  applications must still be rewritten.

\fi

We make the following contributions:
\begin{myitemize4}
    \item A presentation of a deterministic Linux kernel heavily based on that
    of the Determinator kernel. This is the first known adaptation of
    Determinator's kernel design in a widely deployed operating system.
    \item A deterministic high level user library for use by application
    programmers. This library is motivated by Determinator's user library and
    the usefulness of developing programs using a familiar API like the standard
    C library.
    \item An improvement of Determinator's in-memory file system. Our file
    system is modeled off of the BSD Fast File System~\cite{mckusick1984fast},
    and it provides persistence.
    \item We evaluate the performance of deterministic Linux against traditional
    nondeterministic parallel Linux. We also demonstrate a case study of the
    benefits of determinism.
\end{myitemize4}

\iffalse

# Contributions
* We make the following contributions:
* The first known adaptation of Determinator's design to a widely deployed OS.
* A high-level user library for application developers, similar to
  Determinator's user library.
* An improvement of Determinator's in-memory file system based on BSD Fast
  file system.
* An evaluation of deterministic Linux against legacy nondeterministic Linux and
  a case study of deterministic execution's benefits.

\fi

Aviram et al. were motivated by meeting the ``software development, debugging,
and security challenges'' of writing future parallel
applications~\cite{Aviram10}. Determinator was a huge step towards this.
Unfortunately, Determinator has limited uptake outside the academic community.
Linux is a mature and widely deployed operating system available for desktops,
servers, mobile, and embedded systems. Adding determinism alongside
nondeterministic Linux will be a huge next step. If we are lucky, we might be
able to influence how future parallel applications are written.

\iffalse

# Why?
* Aviram et al. were motivated by meeting the "software development, debugging,
  and security challenges" for parallel applications of the future. Determinator
  was a step towards this.
* Determinator itself has limited uptake, since it was written from scratch in
  an academic environment.
* Linux is a mature and widely deployed operating system; available for
  desktops, servers, mobile, embedded.
* If we can add determinism alongside nondeterministic Linux, this will be a
  huge next step.

\fi

\endinput

# Benefits
* According to Bergan et al., determinism provides benefits in four main areas:
  debugging, testing, fault tolerance, and security.
* Repeatability makes debugging easier and can be exploited to provide fault
  tolerance via replication.
* Determinism eliminates the exponential blow up of scheduling sequences.
* Determinism eliminates covert timing channels that can be used to extract
  sensitive data from privileged threads.
* Predictability allows better testing methodologies.

(will elaborate on benefits in separate section)

[MW: state the benefits in one or two sentences (four is too many), and
say that you'll elaborate later.]


