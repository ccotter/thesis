
\section{Related Work}

Many systems attempt to alleviate and control the effects of nondeterminism.
The Velodrome~\cite{flanagan2008velodrome} and
SingleTrack~\cite{sadowski2009singletrack} tools dynamically check atomicity and
determinism constraints in user programs to report possible race conditions or
nondeterministic behavior. Replay debuggers like Instant
Replay~\cite{leblanc1987debugging} record all relevant inputs and thread
interactions to execute a program in a repeatable fashion for debugging
purposes. However, these tools are expensive in terms of run time overhead and
storage. None of these tools do anything to fix the inherently
nondeterministic environment in which developers write programs.

Some systems provide determinism at the expense of using nonstandard programming
languages, compilers, or hardware.
Bocchino et al.'s DPJ extends Java with a deterministic type
system~\cite{bocchino2009type}. Environments like RCDC~\cite{devietti2011rcdc}
and DMP~\cite{Devietti09} provide determinism with transactional memory hardware
support.
CoreDet~\cite{bergan2010coredet} builds on DMP by using the LLVM
compiler~\cite{lattner2004llvm} and a runtime to provide determinism.

CoreDet, DMP, and Grace~\cite{berger2009grace} provide user space deterministic
schedulers for C/C++ programs. However, wild pointer writes can corrupt the
scheduler. Kendo~\cite{olszewski2009kendo} provides a relaxed determinism model:
programs that correctly protect critical sections of code execute
deterministically by synthesizing an arbitrary lock acquisition order based on
instruction counting. Unprotected access to shared variables, however, will lead
to nondeterministic traces, owing to data races. None of these systems are able
to enforce determinism on \emph{arbitrary} user programs as Determinator does.

dOS~\cite{bergan2010deterministic} adds \emph{deterministic process groups}
(DPGs) to the Linux kernel. A deterministic scheduler in the kernel enforces
deterministic thread interactions within a DPG.
To interact with external objects (e.g. network packets), DPGs use a user
space service called a \emph{shim}. The shim gives DPGs complete control over
external nondeterministic inputs, similar to how Determinator's root space
controls external I/O. dOS does not use a "clean-slate" approach as Determinator
does~\cite{Aviram10}, and thus has better backwards compatibility with Linux.
However, like the other deterministic environments mentioned, dOS merely masks
the effects of nondeterminism instead of removing nondeterminism entirely as
Determinator does with its novel programming model.

\endinput

