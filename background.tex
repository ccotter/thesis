
\section{Background}

Computer architects are under increasing pressure to produce multicore
processors: the doubling of transistors every 18 months, or Moore's law, no
longer implies increasing processor clock speeds. Rather, processors
increasingly have more cores, with clock rates holding
steady~\cite{bandwidth2005multicore}. The multicore revolution is shifting
program development from a sequential to parallel paradigm.

Many parallel applications are written in the conventional nondeterministic
threading model. Features like shared memory state are attractive, since they
facilitate easy application development. Unfortunately, writing correct parallel
programs in this environment is hard, particularly due to nondeterminism.

Anyone who has written a multithreaded
application using the conventional threading model is familiar with the
drawbacks. Data races, both low and especially high level, introduce bugs that
often must be solved with difficult to reason about lock abstractions.
Nondeterministic, or even arbitrarily deterministic, scheduling of threads again
introduces bugs and complicates reasoning about the program output.

As more applications become parallel, we do not want a degradation in code
quality. Lee points out problems with the conventional threading model, and he
would like to do away with it~\cite{lee2006problem}. What we would like,
then, is a new programming model that limits or eliminates the potential for
writing programs of poor quality. Bocchino et al. make the case that all
parallel programs should be written using a mode that is ``deterministic by
default''~\cite{bocchino2009parallel}.

\iffalse
\paragraph{Determinism}
\fi
A program's output is a function of its inputs, explicit and implicit.
We say an input is explicit if it is semantically relevant to the program's
output. We consider an input that is irrelevant to the program's intended goal,
but nonetheless influences program output, to be implicit. In most cases,
implicit inputs are random, arbitrary, and uncontrollable; Timing dependencies,
quantum size, and cache size are examples~\cite{Bergan11}.

Since implicit inputs by definition are irrelevant, we would like programs to be
functions of only explicit inputs. We call such programs \emph{deterministic}.
Running a deterministic program on the same input will always generate the same
output, regardless of implicit inputs. Programs whose output depends on
implicit inputs are considered nondeterministic.

Olszewski et al. characterize determinism into two categories: strong and weak
determinism~\cite{olszewski2009kendo}. \emph{Strong determinism} guarantees a
deterministic order of memory operations for a fixed program input and thus
always provides deterministic execution. \emph{Weak determinism} only guarantees
a deterministic order of lock acquisitions. Weak determinism only provides
deterministic execution when a program is written free of data races.

\iffalse
\paragraph{A New Programming Model}
\fi

\iffalse

Parallel programming languages, like DPJ~\cite{bocchino2009type}, provide
deterministic execution; however, since these languages require rewriting
existing programs and have limited potential for wide uptake, we shall not
consider such systems. We also will not consider systems that require
specialized hardware, again due to their limited availability. We would like
a deterministic environment that runs on standard computer hardware written in
popular, convention programming languages like C.

We focus on systems that run programs deterministically written in conventional
languages, like C. Kendo is a weakly deterministic runtime
library~\cite{olszewski2009kendo} for C++ applications.
CoreDet\~cite{bergan2010coredet} provides strong determinism via a compiler
and runtime library.

Even though we consider some inputs to be nondeterministic (like
{\tt gettimeofday()}), these inputs are often semantically relevant to a
program. Thus, we focus our efforts on nondeterministic implicit inputs.

Explicit inputs are what program

\fi

\subsection{Motivation}

Determinism provides many benefits to application
developers~\cite{Bergan11,olszewski2009kendo,bocchino2009parallel}. Bergan et
al. suggest there are four main benefits in
the following areas: debugging, fault tolerance, testing, and security.

\paragraph{Debugging} Debugging multithreaded programs can be difficult since
often bugs are not easily reproducible and tools such as {\tt gdb} are not
always useful for tracking down heisenbugs~\cite{Musuvathi08}. Finding a bug's
root cause becomes easier when a program's execution can be replayed over and
over. Deterministic execution naturally provides replay debugging as a benefit.

\paragraph{Fault tolerance} Fault tolerance through replication obviously relies
on the assumption that running a program multiple times will always return the
same output. Determinism again provides this benefit naturally.

\paragraph{Testing} The difficulties in testing multithreaded applications are
compounded by racy nondeterministic scheduling. Developers and automated test
systems must consider the exponential blow up of possible scheduling sequences.
Determinism helps alleviate this problem by guaranteeing a one-to-one
correspondence between input and output. For each input, there is exactly one
possible logical scheduling sequence of threads. Bergan et al. notes that this
observation can help in designing test strategies~\cite{Bergan11}.

Deterministic execution by itself it not as useful as the stronger guarantee of
\emph{predictability}. Some deterministic environments provide synthesized and
arbitrary schedulers~\cite{Aviram10,Devietti09}. These models do not allow one
to reason about a program and determine the output beforehand, which makes
designing tests difficult. The stronger claim of predictability allows for
developers to design test suites that can accurately check a program's output.
\iffalse
need I say more? tie into Determinator?
\fi

\paragraph{Security} Processes sharing a CPU or other hardware should be
conscious about leaking sensitive data. Covert timing channels can be exploited
by a malicious thread to extract sensitive data from other
threads~\cite{Aviram10cloud}. Determinism eliminates covert timing channels,
since a program is purely a function of explicit inputs and cannot possible rely
on the timings of hardware operations.
\\

There do
exist so called ``point solutions'' that solve problems in single areas at once.
Record and replay debuggers, like Leblanc et al.'s {\tt Instant Replay} system,
aid in debugging parallel programs by logging scheduling sequences and
other relevant interactions in order to replay an execution sequence exactly.
However, these debuggers are costly in terms of storage and performance.
Moreover, these ``point solutions...do not compose well with one
another''~\cite{Bergan11}. Determinism provides benefits in all four areas at
once with a single mechanism.

\iffalse
\begin{itemize}
	\item Deterministic execution benefits, four main areas
	\begin{itemize}
		\item Debugging becomes easier, since bugs are always reproducible. Benefits
			enhanced by Determinator’s predictability.
		\item Testing: one-to-one mapping for inputs to outputs. Again, predictability
			and modularity can simplify designing tests.
		\item Fault tolerance
		\item Security: covert channels
	\end{itemize}
	\item "Point solutions" in particular areas are unrelated to each other and do
		not compose well. (Find point solutions to go into relevant work.)
	\item Determinism is solution offering benefits in all areas at once.
	\item Determinator’s design also provides predictability. Programmers can
		reason about code without having to make assumptions: nothing is arbitrary.
		"Program logic alone" determines how a program proceeds.
	\item Linux is widely used, deployed on millions of machines. Potential for
		uptake is very high if we can make the implementation reasonably easy to
		patch.
	\item Aviram compared Determinator to Linux, but we can compare deterministic
		Linux to nondeterministic Linux for nearly optimal evaluation of this design.
\end{itemize}
\fi

\subsection{Determinator}

Aviram et al. set out to provide
\begin{quote}
a parallel environment that:
(a) is ``deterministic by default,'' except when
we inject nondeterminism explicitly via external inputs;
(b) introduces no data races, either at the memory access level
or at higher semantic levels; (c)
can enforce determinism on arbitrary, compromised or
malicious code for security reasons; and (d) is efficient
enough to use for ``normal-case'' execution of deployed
code, not just for instrumentation during development. \cite{Aviram10}
\end{quote}

To this end, they present Determinator, a novel OS written from the ground up.
For most of the remainder of this section, we will give details Aviram et al.'s
work and contributions; first we will discuss aspects that influenced
Determinator's design. Then, we will look at the actual kernel design itself
and the accompanying user library.

The primary cause of nondeterminism is data races introduced by timing
dependencies. Each source must be accounted for in designing a deterministic
programming model, and we discuss them here.

\paragraph{Explicit Nondeterminism}
Often, programs rely on nondeterministic inputs such as network packets, user
input, or clock time. These inputs are essential to a program being useful;
therefore, a deterministic programming model must incorporate these inputs while
still enforcing determinism. Determinator addresses these ``semantically
relevant'' inputs by turning them into explicit I/O~\cite{Aviram10}.
Applications have complete control over these input sources and can even log the
inputs for reply debugging.

\paragraph{Shared program state}
Traditional multithread programming models provide shared state: threads using
the pthreads API share the entire memory state, and Linux's file system is
shared by all running programs. Data races and incorrect synchronization lead
to nondeterministic execution traces and often introduce unpredictable bugs

Determinator eliminates data races caused by shared program state by eliminating
shared state altogether. Threads operate using a private workspace model and
synchronize program state at explicitly defined program points. When two or more
threads begin executing, each has identical private virtual memory images.
Writes to memory are not visible to other threads until the threads synchronize.

\paragraph{Nondeterministic scheduling abstractions}
Traditional multithreaded synchronization abstractions are often not
deterministic or predictable. Random hardware races determine the next thread to
acquire a mutex lock, and as mentioned before this has debugging and testing
implications. Even though we can record lock acquisition sequences to replay
program execution or use some arbitrary device to choose a deterministic
sequence, the order of acquisition is not predictable. Determinator restricts
itself to only allow naturally deterministic and predictable synchronization
abstractions, such as fork-join.

\paragraph{Globally shared namespaces}
Operating systems introduce nondeterminism by using namespaces that are shared
by the entire system. Process IDs returned by {\tt fork()} and files created
by {\tt mktemp()} are examples. Since these identifiers are nondeterministic,
and only the resource itself, not the identifier, is important for the
application, Determinator does not allow the system to choose resource
identifiers from globally shared namespaces. Since the identifier themselves
are not relevant to an application, applications themselves choose identifiers
deterministically.

\subsection{The Determinator Kernel}

Determinator organizes processes in a nested process model~\cite{Ford96}.
Processes cannot outlive their parents and can only communicate with their
parents and children. In line with the earlier discussion of nondeterminism,
the kernel ``provides no file systems, writable shared memory, or other
abstractions that imply globally shared state''~\cite{Aviram10}. Only ``the
distinguished root [process] has direct access to nondeterministic
inputs''~\cite{Aviram10}. All other processes must communicate directly or
indirectly with the root process to access I/O devices.

\begin{center}
\begin{table*}[t]
\centering
\begin{tabular}{c | c | l}
Call & Interacts with & Description \\
\hline
Put & Child & Copy register state and/or virtual memory range into child, and optionally start child executing. \\
Get & Child & Copy register state, virtual memory range, and/or changes since the last snapshot out of a child. \\
Ret & Parent & Stop and wait for parent to issue a Get or Put. Processor traps also cause implicit Ret. \\
\end{tabular}
\caption{System calls comprising Determinator’s kernel API.}
\label{tab:syscalls}
\end{table*}
\end{center}

\begin{center}
\begin{table}[t]
\begin{tabular}{c | c | c | l}
Put & Get & Option & Description \\
\hline
X & X & Regs & PUT/GET child’s register state. \\
X & X & Copy & Copy memory to/from child. \\
X & X & Zero & Zero-fill virtual memory range. \\
X &  & Snap & Snapshot child’s virtual memory. \\
X &  & Start & Start child space executing. \\
 & X & Merge & Merge child’s changes into parent. \\
X & X & Perm & Set memory access permissions. \\
X & X & Tree & Copy (grand)child subtree. \\
\end{tabular}
\caption{Options/arguments to the Put and Get calls.}
\label{tab:options}
\end{table}
\end{center}

\paragraph{Kernel Interface} Processes communicate with the kernel via three
syscalls summarized in Table \ref{tab:syscalls}: {\tt Put}, {\tt Get}, and
{\tt Ret}. {\tt Put} and {\tt Get} take parameters that specify various
operations outlined in Table \ref{tab:options}.

...obviously needs more content

\subsection{Deterministic Linux}

With Determinator's design presented, we can now motivate a deterministic
Linux.
Determinator was written from scratch in an academic environment with
determinism as the main OS design goal. In some sense, Determinator is not a
\emph{real} operating system, and the potential uptake outside the academic
world is minimal. On the other hand, Linux is a mature and widely deployed
nondeterministic operating system. Linux is installed on millions of systems
including desktop computers, embedded systems, and mobile devices. In other
words, Linux is a \emph{real} operating system used in the real world. By
adding determinism to Linux, we are able to take advantage of the widespread
use and adoption of Linux; the potential userbase for a deterministic LInux is
much greater than that of Determinator. Furthermroe, we can evaluate
Determinator's design in a real operating system.

\endinput

