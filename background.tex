
\section{Background}

Computer architects are under increasing pressure to produce multicore
processors: the doubling of transistors every 18 months, or Moore's law, no
longer implies increasing processor clock speeds. Rather, processors
increasingly have more cores, with clock rates holding
steady~\cite{bandwidth2005multicore}. The multicore revolution is shifting
program development from a sequential to parallel paradigm.

Many parallel applications are written in the conventional nondeterministic
threading model. Features like shared memory state are attractive, since they
facilitate easy application development. Unfortunately, writing correct parallel
programs in this environment is hard, particularly due to nondeterminism.

Anyone who has written a multithreaded
application using the conventional threading model is familiar with the
drawbacks. Data races, both low and especially high level, introduce bugs that
often must be solved with difficult to reason about lock abstractions.
Nondeterministic, or even arbitrarily deterministic, scheduling of threads again
introduces bugs and complicates reasoning about the program output.

Lee points out problems with the conventional threading model, and he
would like to do away with it~\cite{lee2006problem}. As more applications
become parallel, we do not want a degradation in code quality. What we would
like, then, is a new programming model that limits or eliminates the possibility
of buggy writing programs. In search of a solution, Bocchino et al. make the
case that all parallel programs should be written using a programming model that
is ``deterministic by default''~\cite{bocchino2009parallel}.

\iffalse
\paragraph{Determinism}
\fi
In general, a program's output is a function of its inputs, both explicit and
implicit. We say an input is explicit if it is semantically relevant to the
program's output. We consider an input that is irrelevant to the program's
intended goal, but nonetheless influences program output, to be implicit. In
most cases, implicit inputs are random, arbitrary, and uncontrollable; timing
dependencies, quantum size, and cache size are examples~\cite{Bergan11}.

Since implicit inputs by definition are irrelevant, we would like programs to be
functions of only explicit inputs. We call such programs \emph{deterministic}.
Running a deterministic program on the same input will always generate the same
output, regardless of implicit inputs. Programs whose output depends on
implicit inputs are considered nondeterministic.

Olszewski et al. characterize determinism into two categories: strong and
weak~\cite{olszewski2009kendo}. \emph{Strong determinism} guarantees a
deterministic order of memory operations for a fixed program input and thus
always provides deterministic execution. \emph{Weak determinism} only guarantees
a deterministic order of lock acquisitions. Weak determinism provides
deterministic execution only when a program is written free of data races. In
the presence of racy data unprotected by a lock, weakly deterministic systems
fail to execute deterministically.

\iffalse
\paragraph{A New Programming Model}
\fi

\iffalse

Parallel programming languages, like DPJ~\cite{bocchino2009type}, provide
deterministic execution; however, since these languages require rewriting
existing programs and have limited potential for wide uptake, we shall not
consider such systems. We also will not consider systems that require
specialized hardware like DMP~\cite{Devietti09}.
We would like a deterministic environment that runs on standard
computer hardware written in popular, convention programming languages like C.

We focus on systems that run programs deterministically written in conventional
languages, like C. Kendo is a weakly deterministic runtime
library~\cite{olszewski2009kendo} for C++ applications.
CoreDet\~cite{bergan2010coredet} provides strong determinism via a compiler
and runtime library.

Even though we consider some inputs to be nondeterministic (like
{\tt gettimeofday()}), these inputs are often semantically relevant to a
program. Thus, we focus our efforts on nondeterministic implicit inputs.

\fi

\subsection{Motivation for Determinism}

Determinism provides many benefits to application
developers~\cite{Bergan11,olszewski2009kendo,bocchino2009parallel}. Bergan et
al. suggest there are four main benefits in
the following areas: debugging, fault tolerance, testing, and security.

\paragraph{Debugging} Debugging multithreaded programs can be difficult, because
often bugs are not easily reproducible, and tools such as {\tt gdb} are not
always useful for tracking down heisenbugs~\cite{Musuvathi08}. Finding a bug's
root cause becomes easier when a program's execution can be replayed over and
over. Deterministic execution naturally provides replay debugging as a benefit.

\paragraph{Fault tolerance} Fault tolerance through replication obviously relies
on the assumption that running a program multiple times will always return the
same output. Determinism again provides this benefit naturally.

\paragraph{Testing} The difficulties in testing multithreaded applications are
compounded by racy nondeterministic scheduling. Developers and automated test
systems must consider the exponential blow up of possible scheduling sequences.
Determinism helps alleviate this problem by guaranteeing a one-to-one
correspondence between input and output. For each input, there is exactly one
possible logical scheduling sequence of threads. This observation eliminates the
need to consider what scheduling interactions can occur, and ultimately helps
developers design test strategies~\cite{Bergan11}.

In addition to explicit inputs, schedule sequences are affected by implicit
inputs like quantum size, data races, and memory ownership granularity.
Deterministic systems like Kendo~\cite{olszewski2009kendo} and
DMP~\cite{Devietti09} are especially dependent on quantum size~\cite{Bergan11}.
In other words, these systems provide deterministic execution, but the
scheduling sequence is still a function of implicit inputs.

On the other hand, Determinator's programming model provides something stronger
than determinism: \emph{predictability}. Whereas systems like DMP seek to
``control, detect, or reproduce'' data races, Determinator's programming model
is \emph{naturally deterministic}: it avoids ``introducing data races or other
nondeterministic behavior in the first place''~\cite{Aviram10}. With
predictability, programmers can reason about every relevant aspect of a program,
including scheduling. Determinism, and the stronger property of predictability,
thus make testing applications easier.

\paragraph{Security} Processes sharing a CPU or other hardware should be
conscious about leaking sensitive data. A malicious thread can exploit covert
timing channels to extract sensitive data from other, perhaps privileged,
threads~\cite{Aviram10cloud}. Determinism eliminates covert timing channels,
since a program is purely a function of explicit inputs and cannot possible rely
subtly on the timings of hardware operations.
\\

To further motivate determinism, we consider systems that solve that above
problems. So called ``point solutions'' solve in problems in single areas at
once. Record and replay debuggers, like Leblanc et al.'s {\tt Instant Replay}
system, aid in debugging parallel programs by logging scheduling sequences and
other relevant interactions in order to replay an execution sequence exactly.
However, these debuggers are costly in terms of storage and performance.
\iffalse Even with replay ability, the execution sequence is still arbitrary and
gives the programmer no intuition about the program's behavior. \fi
In general, these ``point solutions...do not compose well with one
another''~\cite{Bergan11}. On the other hand, determinism provides benefits in
all four areas at once with a single mechanism.

\iffalse
\begin{itemize}
	\item Deterministic execution benefits, four main areas
	\begin{itemize}
		\item Debugging becomes easier, since bugs are always reproducible. Benefits
			enhanced by Determinator’s predictability.
		\item Testing: one-to-one mapping for inputs to outputs. Again, predictability
			and modularity can simplify designing tests.
		\item Fault tolerance
		\item Security: covert channels
	\end{itemize}
	\item ``Point solutions'' in particular areas are unrelated to each other and do
		not compose well. (Find point solutions to go into relevant work.)
	\item Determinism is solution offering benefits in all areas at once.
	\item Determinator’s design also provides predictability. Programmers can
		reason about code without having to make assumptions: nothing is arbitrary.
		``Program logic alone'' determines how a program proceeds.
	\item Linux is widely used, deployed on millions of machines. Potential for
		uptake is very high if we can make the implementation reasonably easy to
		patch.
	\item Aviram compared Determinator to Linux, but we can compare deterministic
		Linux to nondeterministic Linux for nearly optimal evaluation of this design.
\end{itemize}
\fi

\subsection{Determinator}

Aviram et al. set out to provide
\begin{quote}
a parallel environment that:
(a) is ``deterministic by default,'' except when
we inject nondeterminism explicitly via external inputs;
(b) introduces no data races, either at the memory access level
or at higher semantic levels; (c)
can enforce determinism on arbitrary, compromised or
malicious code for security reasons; and (d) is efficient
enough to use for ``normal-case'' execution of deployed
code, not just for instrumentation during development. \cite{Aviram10}
\end{quote}

To this end, they presented Determinator, a novel OS written from the ground up.
For most of the remainder of this section, we will recapitulate Aviram et al.'s
work and contributions; first we will discuss aspects that influenced
Determinator's design. Then, we will look at the actual kernel design itself
and the accompanying user library.

The primary cause of nondeterminism is data races introduced by timing
dependencies. Each source of implicit nondeterminism must be accounted for in
designing a deterministic programming model. We discuss them here, and describe
how Determinator handles them.

\paragraph{Explicit Nondeterminism}
Often, programs rely on nondeterministic inputs such as network packets, user
input, or clock time. These inputs are essential to a program being useful;
therefore, a deterministic programming model must incorporate these inputs while
still enforcing determinism. Determinator addresses these ``semantically
relevant'' inputs by turning them into explicit I/O~\cite{Aviram10}.
Applications have complete control over these input sources and can even log the
inputs for reply debugging.

\paragraph{Shared program state}
Traditional multithread programming models provide shared state: threads using
the pthreads API share the entire memory state, and Linux's file system is
shared by all running programs. Data races and incorrect synchronization lead
to nondeterministic execution traces and often introduce unpredictable bugs.

Determinator eliminates data races caused by shared program state by eliminating
shared state altogether. Threads operate using a private workspace model and
synchronize program state at explicitly defined program points. When two or more
threads begin executing, each has identical private virtual memory images.
Writes to memory are not visible to other threads until the threads synchronize.

\paragraph{Nondeterministic scheduling abstractions}
Traditional multithreaded synchronization abstractions are often neither
deterministic nor predictable. Random hardware races determine the next thread to
acquire a mutex lock, and as mentioned before this has debugging and testing
implications. Even though we can record lock acquisition sequences to replay
program execution or use some arbitrary device to choose a deterministic
sequence, the order of acquisition is not predictable. Determinator restricts
itself to only allow naturally deterministic and predictable synchronization
abstractions, such as fork-join.

\paragraph{Globally shared namespaces}
Operating systems introduce nondeterminism by using namespaces that are shared
by the entire system. Process IDs returned by {\tt fork()} and files created
by {\tt mktemp()} are examples. Since these identifiers are nondeterministic,
and only the resource itself, not the identifier, is semantically relevant to
the application, Determinator disallows the system from choosing resource
identifiers from globally shared namespaces. Instead, applications themselves
choose identifiers deterministically.

\subsection{The Determinator Kernel}

Determinator organizes processes in a nested process model~\cite{Ford96}.
Processes cannot outlive their parents and can communicate only with their
parents and children. In line with the earlier discussion of nondeterminism,
the kernel ``provides no file systems, writable shared memory, or other
abstractions that imply globally shared state''~\cite{Aviram10}. Only ``the
distinguished root [process] has direct access to nondeterministic
inputs''~\cite{Aviram10}. It is this root process that can control explicitly
nondeterministic inputs. All other processes must communicate
directly or indirectly with the root process to access I/O devices.

\input{tbl1}
\input{tbl2}

\paragraph{Kernel Interface} Processes communicate with the kernel via three
syscalls, {\tt Put}, {\tt Get}, {\tt Ret}. Table \ref{tab:syscalls} and Table
\ref{tab:options}, reproduced from~\cite{Aviram10}, summarize how the syscalls
work and the options available to {\tt Put} and {\tt Get}.

Determinator enforces a deterministic schedule by requiring programs to
explicitly define synchronization points; this mechanism is described here.
Since the kernel does not manage any global namespaces, user programs specify
a child process ID parameter to {\tt Put} and {\tt Get}. The first {\tt Put}
syscall with a previously unused child ID creates a new child process.
{\tt Put} calls can start a child's execution, and the child will continue to
execute until it invokes {\tt Ret} or generates a processor exception (e.g.
divide by zero). {\tt Put} and {\tt Get} calls block until the child process
stops.

Process state is composed of register state and its entire virtual memory.
The Regs option copies register state from a parent to child or vice versa.
Determinator provides more sophisticated virtual memory options: the Zero
option zeros a virtual memory region in a child; the Copy option copies
virtual memory regions between a parent and its child; the snapshot-merge
mechanism is similar to Copy, but more complicated.

Snap copies the calling process's entire virtual memory state into the specified
child. Invoking {\tt Get} with Merge copies bytes
from the child that have changed since the previous Snap invocation into the
parent. Bytes that changed in the parent but not the child are not copied. Bytes
that changed in both the parent and child generate a \emph{merge conflict}.
The kernel implements Merge efficiently by examining page table entires.

Aviram et al. conclude their discussion of Determinator's kernel by mentioning
the three syscall ``primitives reduce to blocking, one-to-one message channels,
making the space hierarchy a deterministic Kahn
network''~\cite{kahn1974semantics}.

\subsection{Determinator's User Library}

The Determinator kernel alone is enough to enforce deterministic program
execution; however, Aviram et al. provide a high-level user library to make
writing deterministic programs easier and more natural. In this section, we will
go over
the five main areas discussed by Aviram et al. in their ``Emulating High-Level
Abstractions'' section: process API, file
system, I/O, shared memory multithreading, and legacy thread APIs~\cite{Aviram10}.

\paragraph{Process API}
Determinator provides an interface similar to that of {\tt fork/wait}.
All of these functions are implemented in user space instead of kernel space.
To do a {\tt fork()}, the parent does a single {\tt dput()} to copy its register
and memory into a child. The user library must manage a list of ``free'' process
ID numbers, since the user must specify the child process ID. A {\tt waitpid()}
works by entering a loop querying the status of the child; if the child needs
more input to continue running (through a mechanism described below), it
sets its status appropriately and issues a {\tt dret()}. The parent gives the
child more input and sets it in motion again. Once the child finishes executing,
it marks itself done and issues a {\tt dret()}.

\paragraph{File System}
Since processes can only access their register set and memory, Determinator
provides an in-memory file system. Each process has a private copy of the
system's file system. A {\tt fork()} copies the parent's file system state into
the child. The parent and child work on private copies of the file system.

At synchronization points, the parent reconciles the changes using file
versioning techniques~\cite{parker1983detection}. Two files may not be
concurrently modified; such cases lead to a reconciliation conflict. The parent
and child may, however, perform \emph{append-only} changes to the same file.
The file is reconciled by appending the child's additions to the end of the
parent's file, and vice-versa.

The file system is limited, however. The total file system size is limited
by the process's address space; on 32-bit systems, this is a serious limitation.
Since the file system resides in virtual memory, buggy programs can write to
the memory where the file system resides, corrupting the file system. Lastly,
the file system only supports up to 256 files, each with a max of 4MB in
size~\cite{Aviram10cloud}.

\paragraph{I/O}
Since Deterministic processes have no access to external I/O, Determinator
emulates I/O as a special case of the file system. Library functions like
{\tt getchar()} and {\tt printf()} read and write from special files
{\tt stdin} and {\tt stdout}, respectively.

A {\tt printf()} appends output to {\tt stdout}. At program synchronization
points, the file system is merged. {\tt stdout} output is forwarded
to the parent and eventually reaches the root process where the root can
actually write the output to the system's I/O device.
A program does a {\tt read()} to obtain the next unread character(s) in
{\tt stdin}. If the file is out of unread characters, {\tt read()} does a
{\tt dret()} to ask for more input from the parent.

Since the file system supports ``append-only'' conflicts, the above strategy
works for handling I/O. As all processes reconcile their file systems, each
process will see all other process's {\tt printf()}s.

\paragraph{Legacy Multithreading APIs}
Determinator can emulate shared memory multithreading and other legacy
multithreaded APIs like pthreads. However, we will not discuss either here,
since we do not use these techniques in deterministic Linux. However, we note
that since Determinator emulates these legacy thread APIs using its three
syscall interface, deterministic Linux could support these APIs.

\subsection{Deterministic Linux}

With Determinator presented, we can now motivate a deterministic Linux.
Determinator was written from scratch in an academic environment with
determinism as the main OS design goal. In some sense, Determinator is not a
\emph{real} operating system. The potential uptake outside the academic
world is minimal. On the other hand, Linux is a mature and widely deployed
nondeterministic operating system. Linux is installed on millions of systems
including desktop computers, embedded systems, and mobile devices. In other
words, Linux is a real operating system used in the real world, and by adding
determinism to Linux, we are able to take advantage of its widespread
use and adoption; the potential userbase for a deterministic Linux is
much greater than that of Determinator.

Adding an inherently deterministic and predictable programming model to Linux is
a huge step in attempting to influencing how developers write the parallel
applications of the future.

\endinput

